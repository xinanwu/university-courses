
\documentclass[a4 paper,12pt]{article}
\usepackage[inner=2.0cm,outer=2.0cm,top=2.5cm,bottom=2.5cm]{geometry}
\usepackage{setspace}
\usepackage[rgb]{xcolor}
\usepackage{tabu}
\usepackage{pifont}
\usepackage{multirow}
\usepackage{longtable}
\usepackage{graphicx}
\usepackage{verbatim}
\usepackage{longtable}
\usepackage{subcaption}
\usepackage{fancyhdr}
\usepackage[colorlinks=true, urlcolor=blue, linkcolor=blue, citecolor=blue]{hyperref}
\usepackage{booktabs}
\usepackage{amsmath,amsfonts,amsthm,amssymb}
\usepackage{setspace}
\usepackage{listings}
\usepackage{fancyhdr}
\usepackage{lastpage}
\usepackage{tikz}
\usetikzlibrary{positioning, arrows.meta}
\usepackage{extramarks}
\usepackage{ctex,amsmath,amsfonts,amssymb,bm,hyperref,graphicx}
%\lstset{
	%	commentstyle=\color{red!50!green!50!blue!50},%代码块背景色为浅灰色
	%	rulesepcolor= \color{gray}, %代码块边框颜色
	%	breaklines=true,  %代码过长则换行
	%	numbers=left, %行号在左侧显示
	%	numberstyle= \small,%行号字体
	%	keywordstyle= \color{blue},%关键字颜色
	%	frame=shadowbox,%用方框框住代码块
	%	basicstyle=\ttfamily
	%}
\definecolor{dkgreen}{rgb}{0,0.6,0}
\definecolor{mauve}{rgb}{0.9,0.1,0.4}
\definecolor{ash}{rgb}{0.8,0.8,0.8}
\lstset{ 
	language=Octave,                % the language of the code
	basicstyle=\ttfamily,           % the size of the fonts that are used for the code
	numbers=left,                   % where to put the line-numbers
	numberstyle=\small\color{gray},  % the style that is used for the line-numbers
	stepnumber=1,                   % the step between two line-numbers. If it's 1, each line
	% will be numbered
	numbersep=5pt,                  % how far the line-numbers are from the code
	backgroundcolor=\color{ash},      % choose the background color. You must add \usepackage{color}
	rulesepcolor= \color{gray}, %代码块边框颜色
	showspaces=false,               % show spaces adding particular underscores
	showstringspaces=false,         % underline spaces within strings
	showtabs=false,                 % show tabs within strings adding particular underscores
	frame=single,                   % adds a frame around the code
	rulecolor=\color{black},        % if not set, the frame-color may be changed on line-breaks within not-black text (e.g. commens (green here))
	tabsize=2,                      % sets default tabsize to 2 spaces
	captionpos=b,                   % sets the caption-position to bottom
	breaklines=true,                % sets automatic line breaking
	breakatwhitespace=false,        % sets if automatic breaks should only happen at whitespace
	title=\lstname,                   % show the filename of files included with \lstinputlisting;
	% also try caption instead of title
	frame=shadowbox,%用方框框住代码块
	keywordstyle=\color{blue},          % keyword style
	commentstyle=\color{dkgreen},       % comment style
	stringstyle=\color{mauve},         % string literal style
	escapeinside={\%*}{*)},            % if you want to add LaTeX within your code
	morekeywords={*,...}               % if you want to add more keywords to the set
}
\usepackage{chngpage}
\usepackage{soul,color}
\usepackage{graphicx,float,wrapfig}
\newcommand{\homework}[3]{
	\pagestyle{myheadings}
	\thispagestyle{plain}
	\newpage
	\setcounter{page}{1}
	\noindent
	\begin{center}
		\framebox{
			\vbox{\vspace{2mm}
				\hbox to 6.28in { {\bf Microeconomics \hfill} {\hfill {\rm #2} {\rm #3}} }
				\vspace{4mm}
				\hbox to 6.28in { {\Large \hfill #1  \hfill} }
				\vspace{3mm}}
		}
	\end{center}
	\vspace*{4mm}
}
\newcommand\numberthis{\addtocounter{equation}{1}\tag{\theequation}}
\renewcommand\contentsname{Contents}
\begin{document}
	\homework{Homework03}{Group45}{吴熙楠}
	\tableofcontents
	\newpage
\section{Problem\quad 1}
\noindent
\textbf{Answer:}
\par(a)$MU_{x}=\dfrac{\partial u(x,y)}{\partial x}=3,MU_{y}=\dfrac{\partial u(x,y)}{\partial y}=1$
\par (b)Yes, because $MU_{x}=3>0$ and $MU_{y}=1>0$
\par (c)$MRS_{x,y}=-\dfrac{MU_{x}}{MU_{y}}=-3$, it means that having one book for his friends is as good as having three for himself. In other words, the utility keeps the same while he has 3 more y and 1 less x.
\par (d)No, because MRS is constant no matter what x and y are.
\section{Problem\quad 2}
\noindent
\textbf{Answer:}
\par (a)We suppose that his utility function is $Eu=\pi u(w_{g})+(1-\pi )u(w_{b})$\\$=\pi u(w+xr_{g})+(1-\pi)u(w+xr_{b})$
\par $\dfrac{dEu}{dx}=\pi r_{g}\dfrac{du(z)}{dz}|_{z=w+xr_{g}}+(1-\pi)r_{b}\dfrac{du(z)}{dz}|_{z=w+xr_{b}}$
\par \ding{172} for $0\le x\le w$, if $\dfrac{dEu}{dx}>0$, then the more he invests, the more utility he can get.
\par The best strategy is that he invests all his wealth($x=w$).
\par \ding{173} for $0\le x\le w$, if $\dfrac{dEu}{dx}<0$, then the more he invests, the less utility he can get.
\par The best strategy is that he invests zero($x=0$). 
\par \ding{174} for $0\le x\le w$, if $\dfrac{dEu}{dx}==0$, then no matter what he invests, the utility is a constant.
\par Then no matter what he invests, the utility is a constant. 
\par \ding{175}for $0\le x\le w$, if Eu increases first but decreases later.
\par $\dfrac{dEu}{dx}=
		\bigg\lbrace
		\begin{aligned}\,
			& >0, \,
			&&\text{for}\ \ 0\le x<x_{0}, \\[-.5ex]
			& \le 0, 
			&&\text{for}\ \ x_{0}\le x\le w,
		\end{aligned}$
\par Then for $x=x_{0}$, we can get the best utility. So the best strategy is that he invests $x_{0}$.
\par \ding{176}for $0\le x\le w$, if Eu decreases first but increases later.
\par $\dfrac{dEu}{dx}=
		\bigg\lbrace
		\begin{aligned}\,
			& <0, \,
			&&\text{for}\ \ 0\le x<x_{1}, \\[-.5ex]
			& \ge 0, 
			&&\text{for}\ \ x_{1}\le x\le w,
		\end{aligned}$
\par Then if $Eu(x=w)>Eu(x=0)$, then the best strategy is that he invests $w$; if $Eu(x=w)<Eu(x=0)$, then the best strategy is that he invests $0$; if $Eu(x=w)==Eu(x=0)$, then he can choose to invest 0 or $w$.
\par (b)Similarly we should change the utility:$Eu^{\prime}=\pi u(w_{g})+(1-\pi )u(w_{b})$\\$=\pi u(w+(1-t)xr_{g})+(1-\pi)u(w+(1-t)xr_{b})$
\par $\dfrac{dEu^{\prime}}{dx}=\pi(1-t) r_{g}\dfrac{du(z)}{dz}|_{z=w+(1-t)xr_{g}}+(1-\pi)(1-t)r_{b}\dfrac{du(z)}{dz}|_{z=w+(1-t)xr_{b}}=(1-t)\dfrac{dEu(x^{\prime})}{dx^{\prime}}|_{x^{\prime}=(1-t)x}$
\par \ding{172} for $0\le x\le w$, if $\dfrac{dEu^{\prime}}{dx}>0$, then the more he invests, the more utility he can get.
\par The best strategy is that he invests all his wealth($x=w$).
\par \ding{173} for $0\le x\le w$, if $\dfrac{dEu^{\prime}}{dx}<0$, then the more he invests, the less utility he can get.
\par The best strategy is that he invests zero($x=0$). 
\par \ding{174} for $0\le x\le w$, if $\dfrac{dEu^{\prime}}{dx}==0$, then no matter what he invests, the utility is a constant.
\par Then no matter what he invests, the utility is a constant. 
\par \ding{175}for $0\le x\le w$, if $Eu^{\prime}$ increases first but decreases later.($x_{0}^{\prime}=\dfrac{x_{0}}{1-t},x_{0}<(1-t)w$)
\par $\dfrac{dEu^{\prime}}{dx}=
		\bigg\lbrace
		\begin{aligned}\,
			& >0, \,
			&&\text{for}\ \ 0\le x<x_{0}^{\prime}, \\[-.5ex]
			& \le 0, 
			&&\text{for}\ \ x_{0}^{\prime}\le x\le w,
		\end{aligned}$
\par Then for $x=x_{0}^{\prime}$, we can get the best utility. So the best strategy is that he invests $x_{0}^{\prime}$.
\par \ding{176}for $0\le x\le w$, if $Eu^{\prime}$ decreases first but increases later.($x_{1}^{\prime}=\dfrac{x_{1}}{1-t},x_{1}<(1-t)w$)
\par $\dfrac{dEu^{\prime}}{dx}=
		\bigg\lbrace
		\begin{aligned}\,
			& <0, \,
			&&\text{for}\ \ 0\le x<x_{1}^{\prime}, \\[-.5ex]
			& \ge 0, 
			&&\text{for}\ \ x_{1}^{\prime}\le x\le w,
		\end{aligned}$
\par Then if $Eu^{\prime}(x=w)>Eu^{\prime}(x=0)$, then the best strategy is that he invests $w$; if $Eu^{\prime}(x=w)<Eu^{\prime}(x=0)$, then the best strategy is that he invests $0$; if $Eu^{\prime}(x=w)==Eu^{\prime}(x=0)$, then he can choose to invest 0 or $w$.
\par We should notice that $x_{0}^{\prime}=\dfrac{x_{0}}{1-t},x_{1}^{\prime}=\dfrac{x_{1}}{1-t}$.
\section{Problem\quad 3}
\noindent
\textbf{Answer:}
\par Assume that he borrows x yuan in the first period, then $c_{1}=x$ and $c_{2}=8800-1.1x$.\\
So $u(x)=x(8800-1.1x)$, and $\dfrac{du}{dx}=8800-2.2x$. Because this is a quadratic function, we can set $\dfrac{du}{dx}$ equal to zero and then we can get the maximum utility function.\\
So $\dfrac{du}{dx}=8800-2.2x=0\rightarrow x=c_{1}=4000,c_{2}=4400$, and $u_{max}=17600000$
\section{Problem\quad 4}
\noindent
\textbf{Answer:}
\par (a)$\epsilon_{x,I}=\dfrac{\partial x}{\partial I}\dfrac{I}{x}=\dfrac{2-2I}{p_{x}}\dfrac{p_{x}}{2-I}=\dfrac{2(1-I)}{2-I}$
\par (b)If beef is a normal good, then $0<\epsilon_{x,I}\rightarrow 0<\dfrac{2(1-I)}{2-I}$.
\par So we can solve this inequality, and we can get that $I<1$ or $I>2$.
\par But we also know that $x\ge 0$, so we can get that $0\le I\le 2$.
\par So if $0\le I<1$, the beef is a normal good.
\section{Problem\quad 5}
\noindent
\textbf{Answer:}
\par (a)We can know that $Q^{D}=Q^{S}\rightarrow 150-50p=60+40p$, So we can solve that $p=1,Q=100$.
\par (b)$Q^{S\prime}=60+40(p-0.5)=40+40p$,then $Q^{D}=Q^{S\prime}\rightarrow 150-50p=40+40p$.\\
We can solve that $p^{\prime}=\dfrac{11}{9},Q^{\prime}=\dfrac{800}{9}$
\par(c)$CS_{1}=\int_{0}^{100}(\dfrac{150-q}{50})dq-1\times 100=200-100=100$
\par $CS_{2}=\int_{0}^{\frac{800}{9}}(\dfrac{150-q}{50})dq-\dfrac{11}{9}\times \dfrac{800}{9}=\dfrac{6400}{81}$
\par So $\Delta CS=CS_{2}-CS_{1}=\dfrac{6400}{81}-100=-\dfrac{1700}{81}$
\par (d)$PS_{1}=1\times 100-\int_{0}^{100}(\dfrac{q-60}{40})dq=100+25=125$
\par $PS_{2}=\dfrac{11}{9}\times \dfrac{800}{9}-\int_{0}^{\frac{800}{9}}(\dfrac{q-40}{40})dq=\dfrac{8000}{81}$
\par $\Delta PS=PS_{2}-PS_{1}=\dfrac{8000}{81}-125=-\dfrac{2125}{81}$
\par (e)government benefit=$0.5\times \dfrac{800}{9}=\dfrac{400}{9}$
\par(f)the dead-weight loss=$0.5\times (100-\dfrac{800}{9})\times 0.5=\dfrac{25}{9}$\\
We can find that $\Delta CS+\Delta PS$ + government benefit + the dead-weight loss=0.
\end{document} 
