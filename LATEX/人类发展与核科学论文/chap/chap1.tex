\chapter{背景}
1492年,当哥伦布发现新大陆时,世界人口为4.5亿(相当于今天人口的6.5\%),世界能源消耗还不到今天的1\%。世界人口稀少,能源充足可以支持文明发展。18世纪中叶,蒸汽机的发明开启了工业革命,20世纪初,爱因斯坦的质能方程使人类能够控制原子内部的各种过程,从而发展出各种现代新技术,因此,能源消耗也随之增加,人口也随之增加。目前,世界人口从哥伦布时代的4.5亿增长到70亿,在过去的100年里,世界人口增长了4倍,而能源消耗增长了10倍,即,能源消耗的增长快于人口增长,但是为了满足人们不断增长的物质文化需要,这种人均能源消费增长的趋势必然会继续下去。\cite{Lee2011NuclearFE}
\par 在过去的100年中,世界人口翻倍的时间约为50年,但能源消耗翻倍的时间约为30年。如果这个趋势继续下去,世界将会达到能源供应的临界点。因为能源资源是有限的,据估计供应趋势将在今后短短几十年内达到顶峰。我们需要开发一种新的无限的能源,并且清洁无污染的能源,不会污染环境。这时候我们想起了爱因斯坦的质能方程,核裂变和核聚变将会是一种不错的选择。但是对于核裂变反应,我们将会一直有放射性核废料积累,这显然不是长久之计。而对于核聚变,这与在恒星中发生的过程相同,将会利用质量亏损而产生的巨大的能量,并且因为使用的不是放射性元素,因此不会有核裂变的放射性安全隐患;而用于核聚变的燃料,锂和氘不仅无污染,而且因为海水中有丰富的锂,各种形式的水中也有丰富的氘,原料也十分容易获取。因此,核聚变能量的应用前景是非常优秀的。
