\chapter{总结与展望}
在本篇文章中我们从目前人口增长的大背景出发说明了核聚变这种具有巨大潜力的能源的必要性,可以说是如果我们除非移居到地外行星(然而这种方法实施的难度远大于开发核聚变能的难度),否则地球目前的可利用资源将会很快不能支持人类的需求,因此核聚变能的必要性就凸显了出来,从长远来看,核能也将是继石油、煤和天然气之后的主要能源,人类将从“石油文明”走向“核能文明”。它的优势毫无疑问是巨大的,甚至可以让人类在以后几千万年都不用为了能源问题困扰,但是它也有一些劣势,其中最为重要的一条就是核聚变的发生需要极为严苛的条件,虽然在实验室条件下我们已经可以实现可控核聚变反应堆,但是我们离真正的实用商业化还是有很长一段路要走的;同时,近年来我国在核聚变能源方面的研究与发展迅速,各国之间对于核聚变能源的研究也趋向于合作共赢,因此,可以想象在不久后的将来,我们将会见证核聚变能作为新能源的崛起,人类发展也将进入新的时代。