\chapter{核聚变能的前景}
\section{核聚变能的优势}
\subsection{核聚变释放的能量巨大}
众所周知,在化学反应中,我们在获取能量的过程中,例如石油、天然气燃烧,每涉及到一对原子,只能得到$eV$量级的能量(一个可见光光子能量量级)。这取决于外部原子电子的结合能在$eV$范围内。但是在另一方面,在核转换过程中,我们通过爱因斯坦的质能关系,每一对核所能吸收或释放的能量量为$MeV$。因此,对于每获得一定量的能量,核反应转化的质量大约是燃烧所需质量的百万分之一。因此,对于原子核来说,每一个稳定的原子核的质量都比其所有组成部分(质子和中子)的质量之和要低\cite{Focardi2010ANE}。而相对于核裂变而言核聚变也能产生更多的能量,如果燃料的质量相等,核聚变释放的能量是裂变的3到4倍\cite{Petrescu2017SomePS}。因此在释放能量的“性价比”方面,核聚变显然是目前最具潜力的新能源。
\subsection{核聚变原料充足}
核聚变的原料相对于普通化石燃料而言充足得多,现代普遍估计在全球的化石燃料将在$100\sim 200$年内达到临界值,到时候将会把所有化石燃料消耗光。但是对于核聚变而言,如氘和氚之核聚变反应,其原料可直接取自海水(海洋自然包含这样一个大规模的氘($33g/m^{3}$)),氘在海洋中相当丰富,海洋中每$1km^{3}$海水中氘原子所具有的潜在能量相当于燃烧13600亿桶原油的能量,这个数字约为地球上蕴藏的石油总储量,而且海洋中的锂也可以用来提供核聚变的燃料,理论上用氘和氚进行核聚变可以满足当前的能源消耗人类一亿年的使用,这样几乎可以说是原料无穷无尽了,因此对于核聚变能而言我们几乎不用考虑能源原料问题。
\subsection{核聚变能为清洁安全的能源}
首先,相较于普通化石燃料,核聚变不会产生温室气体,因此也不会造成温室气体带来的全球环境问题;第二,相较于核裂变发电,核聚变产生的核废料半衰期极短(低管理成本、核泄漏时总危害较低)、安全性也更高(不维持对核的约束便会停止反应),而且与核裂变不同(核裂变用作燃料的原材料具有放射性,核裂变产生的能量也对应着放射性废料的产生),核聚变产物本身(主要是$_{2}^{4}He$)没有放射性,但当反应用来释放快中子时,它们可以将捕获它们的原子核转化为同位素,其中一些是放射性的。同时核聚变也是一种中子源,借此可以触发核裂变,这被称为次临界核裂变,次临界核裂变不但安全性接近核聚变,且技术难度较核聚变发电低,还可以处理核裂变发电造成的核废料问题,让这些核废料的半衰期由数万年缩短为数百年,可以说是一举多得。
\section{核聚变能的劣势}
\subsection{核聚变的严苛条件}
虽然看起来从理论上来讲核聚变反应能为人类提供无穷无尽的能量来源,但实际上我们现在还没能做到可以商业化生产的可控核聚变堆,这是因为核聚变反应所需要的严苛的反应条件与技术要求。


首先,在原理部分我们也提到了,需要克服库仑势垒,我们需要外界给一个强力的能量来源,比如我们需要提供$10^{8}K$的超高温(冷核反应由于实验室无法重现故不与讨论),而我们目前的加热效率低下,我们目前主要是等离子体电流加热,但是由于电子与等离子体碰撞的能量损耗太大,且普通导体的允许临界电流密度太小,因此我们用这种方法加热的效率低下,且难以实现。而惯性约束核聚变与磁约束核聚变目前还不是特别稳定,技术不够成熟,距离真正实用还有一段路要走。
\subsection{核聚变的“放射性”}
虽然核聚变能比核裂变能清洁,但是为了易实现核聚变, 需要使用放射性氖,既然用了放射性原料,会发生放射性污染的危险。而且核聚变反应产生的中子可以跟反应装置的墙壁发生核反应,因为这种聚变堆部件承受着极大的负荷,墙壁被中子轰击后会带有“放射性”,用一段时间之后就必须更换,而且换下来的墙壁成了核废料,这就出现如同核裂变堆那样的核废物处置问题。