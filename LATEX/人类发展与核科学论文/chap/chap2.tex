\chapter{核聚变原理}
20世纪初爱因斯坦发现质能方程,人们由此发现了原子核核能的巨大潜力。根据质能方程$E=mc^{2}$,原子核净质量变化将会造成能量的释放。如果有多个重的原子核变化为多个轻的原子核,称为核裂变,如原子弹爆炸;如果是由较轻的原子核变化为较重的原子核,称为核聚变,因为核聚变是给大部分活跃恒星提供能量的过程,因此又称热核反应。如:两个氘发生聚变反应生成一个氢和一个氚,将会放出$4.03MeV$能量,反应方程式为:$_{1}^{2}H+_{1}^{2}H\rightarrow_{1}^{3}H+_{1}^{1}H+4.03MeV$。


核聚变将较轻的原子核结合形成较重的原子核,但是由于原子核带正电,因此原子核之间会有库仑排斥力,这将会阻碍原子核的结合,克服库仑势垒需要巨大的能量。但是由于轻核所带的电荷少,因此轻核之间发生核聚变时需要克服的势垒越小,释放出的能量就越多。



因此如果要进行核聚变反应,首先就必须提高物质的温度,让原子核与电子之间形成等离子体,但由于原子核带正电,彼此间会互相排斥,所以很难使其彼此互相接近,因此我们还必须让等离子体的温度达到超高温($\sim10^{8}K$),还需要达到一定的密度和封闭时间。由于提高物质的温度可以使原子核剧烈运动,因此温度升高,密度变大,封闭的时间越长,彼此接近的机会越大。但由于等离子体很快分散,所以必须先将其封闭。太阳内部是利用巨大万有引力使等离子体封闭,引力场约束形成简并电子气,这在地球上显然是行不通的,当然由于等离子体带电,因此我们可以合理利用磁约束来达到封闭效果。