\documentclass[aps,pre,12pt,preprint,%
	onecolumn,showpacs,showkeys,nofootinbib]{revtex4-2}
\input{expTemplate.tex}
%Miscellaneous
%	\newcommand{\tabindent}{\hspace{2em}}
%FourierTransform
	\newcommand{\fourierf}{\mathscr F}
\begin{document}
%Basic Data
	\title{%
	\texstringonly{\hfil\\[2\baselineskip]}
	\sf\LARGE%
		声光调制锁模激光器%
	\texstringonly{\vspace{3ex}}}
	\author{\fangsong\large%
		吴熙楠%
	\vspace{2mm}}
	\affiliation{\it%
		北京大学物理学院~~学号:\normalfont 1900011413\,}
	\date{\today}
	\keywords{声光调制, 锁模, 拉曼-奈斯衍射}
	\email{xinanwu@pku.edu.cn;}
	
\begin{abstract}
\vspace{10mm}
\begin{spacing}{1.5}\normalsize
\setlength{\parskip}{.3\baselineskip}
%	200—300字,
%	说明用什么方法做了什么事,
%	由此得到什么结果和结论,
%	有何意义. 
%	不用缩略词,不用第一人称.
%	
本实验中尝试利用熔石英制作的声光调制器对 He-Ne 激光器输出的多个纵模进行锁模; 这对于理解声光调制的原理以及对声光锁模技术进行应用具有较大意义.
\end{spacing}
\end{abstract}
\clearpage
\maketitle
\thispagestyle{titlepagestyle}
%
%	\item 课程实验报告应假定读者既不是已知全部实验细节的指导教师,也不是缺少专业知识的公众,而是同领域的实验研究者,或审稿人. 不能要求读者要在读过课程讲义后才能读懂课程实验报告.
%	\item 公式、图和表要分别用阿拉伯数字编列序号. 公式和图表要达到可发表的质量.
%	\item 凡不是自己独立思考得到的内容都应该引参考文献. 不能大段引用同一参考文献. 对复杂问题,应该优先考虑引用参考文献得到结果. 对简单一些的问题才鼓励独立思考.
%	\item 较长的推导和说明可以作为附件提交,不占用报告篇幅.
%	\item 思考题不是报告的组成部分. 应另起一页附在报告的最后.

\newpage
\section{引言}
%	研究论文引言一般包含以下内容:
%	(1)所研究领域背景和现状;
%	(2)有待研究的问题;
%	(3)本研究的目的、主要内容和结果;
%	(4)结果的意义.\par
%	在写实验报告的引言时,同学可以假想自己是第一个做类似研究的人.\par
%	引言一定要切合报告正文,不能漫无目的地介绍背景. 要快速地将读者引导到报告主题上,并作较深入的讨论.\par
%	引言篇幅可以在较大范围内变化,但最长不应超过报告文字篇幅的1/3.\par
%	引言撰写可以参考实验讲义,可以复述,但不能复制讲义上的任何一句话.\par
%%%%%%%%%%%%%%%%%%%%%%%%%%%%%%
锁模技术能够使脉冲激光强度增加并使得脉宽减小. 在上世纪 80 年代后期人们便已经利用锁模技术获得了飞秒量级超短超强脉冲. 这对于非线性光学, 时间分辨光谱学以及等离子体物理等等等学科的发展起了很大的促进作用. 本实验便利用声光调制的方法对 He-Ne 激光器的纵模进行锁模, 从而加深对于锁模技术和声光调制技术的理解, 掌握相关光路的调节方式, 并观察各个因素对于输出脉冲的影响.
\section{理论}
\subsection{锁模激光器原理}
本实验采用的是 He-Ne 激光器, 其相邻纵模圆频率差为$\Delta\omega=\dfrac{\pi c}{L}$; 若激光器介质的
增益线宽为$\Delta\omega_G$,则激光腔内同时有$N=\dfrac{\Delta\omega_G}{\Delta\omega}$个纵模. 而在自由振荡的激光器中 N 个纵模
的初相位之间没有固定的关系, 因而在对时间求平均的过程中激光器总强度正比于各
纵模强度之和. 而若让所有纵模初相位之间建立起固定的联系, 即让它们相干叠加,
则光脉冲的峰值光强会变为自由振荡峰值光强的 N 倍. 并且锁住的纵模个数越多, 锁
模脉宽就越窄.
\par 激光锁模的方法分为被动锁模和主动锁模. 被动锁模具体来说是在激光腔内放入工
作状态不能人为控制的可饱和吸收原件; 主动锁模则是在激光腔内放入可以人为调控
一些参数的调制元件. 本实验中采用主动锁模技术, 具体来说是让激光纵模强度在腔
内受到周期性的调制损耗. 设损耗的函数形式为$\delta=\delta_{0}cos(\Delta\omega t)$, 则受调制损耗的第 q
个纵模振动表示为:$E_{q}(t)=E_{0q}cos(\omega_q t+\phi_q)+0.5E_{0q}((\omega_q+\Delta\omega)t+\phi_q)+0.5E_{0q}((\omega_q-\Delta\omega)t+\phi_q)$
 \par 因而当$\Delta\omega$正好等于
纵模间隔的时候该频率正好与相邻的两个频率的纵模耦合, 因而迫使所有纵模都以相
同的相位振动, 从而实现同步振荡, 达到了锁模的目的.
\subsection{声光调制器原理}
\subsubsection{声光衍射效应}
介质中有超声波传播的时候, 介质会产生应变, 使得折射率发生周期性的变化. 因
而光束通过这个介质之后就会发生衍射. 本实验中采用的声光介质是熔石英, 是各向
同性的介质. 根据入射角的不同和声光作用区长度的不同, 声光衍射可以分为拉曼-奈
斯衍射和布拉格衍射两种. 而根据声波的传播形式也分为声波是行波和驻波两种类型.
本实验中观察到的衍射现象是拉曼-奈斯衍射现象, 因为能看到多级的衍射光.
\subsubsection{驻波型声光器件衍射光强的调制度}
驻波型声光衍射器件各级衍射的光强是被调制的, 随时间演化光强会出现极大和
极小, 定义光强的调制度为$M=\dfrac{I_{max}-I_{min}}{I_{max}}$
. 在接收器的响应时间比调制光波的周期大
很多的时候, 测量的结果反映的是光强的平均值$\bar{I}=0.5(I_{max}+I_{min})$. 对于零级衍射
光强, $I_{max}=J_0^2 (0)$, $I_{min}=J_0^2 (\psi)$, 则可以据此计算出M
;$\psi=\dfrac{2\pi}{\lambda_0}\mu l$为
声制相移, $\mu =\dfrac{1}{2}n^3 pS_0$为折射率变化的振幅. 又声功率$P_a=\dfrac{1}{2}\rho V^3S_0^2 hl$, 可以计算出$\psi$, 从而可以得知声光调制度与声功率的关系.实验中通过测量 0 级衍
射的平均衍射效率可以得知调制度的大小, 从而推算出相应的声功率以及电功率转
化为声功率的效率.

\section{实验装置}
%	在此部分需要将实验条件交待清楚到别人能重复你的实验结果的程度. 此外,还需表明你已尽了最大努力来提高实验精度和结果的可靠性. 简单的不确定度估计可以在此节给出,复杂一些的可以放到分析讨论部分.\par
%	实验条件不仅是指直接影响实验结果的实验参量,而且还包括影响实验质量和可靠性的因素,如室温、空气湿度、基真空、原材料纯度等.\par
%	作为教学实验报告,此节写详细一点没有坏处.\par
%	成段有叙述,必要才分节。
%%%%%%%%%%%%%%%%%%%%%%%%%%%%%%

\section{结果与分析}
%	实验结果应尽量以图表的形式给出. 每一个图表都应该是完整的,即阅读图表时可以不必依赖正文.\par
%	依自己意愿,实验结果和对结果的分析讨论既可分为两节也可合在一节.\par
%	\begin{table}[h]
%	\caption{元件恒流大小,为什么要左对齐呢?奇怪。}
%	\small
%	\begin{tabularx}{.6\linewidth}{C{.3} C{1}}
%	\toprule
%	\midrule
%		元件\footnote{%
%			注释一个看看%
%		} & 恒流大小\footnote{%
%			再开一个!哈哈
%		} \\
%	\midrule
%		Pt  &
%			$\SI{1.00005}{\mA}
%				= \SI{100.005}{\mV} / \SI{100}{\ohm} $ \\
%	\midrule
%	\bottomrule
%	\end{tabularx}
%	\label{tab:ExTab}
%	\end{table}
%	
%	每个图一般包含:图名、轴名、轴、刻度、标尺、数据点、曲线、图例、标注和图注等部分. 应尽量让读者不看正文就能基本理解图的含意.\par
%	逐点测量得到的函数关系要同时用表格和图给出. 需要作比较的多条曲线要画在同一图上.\par
%	为避免读者在图表和正文间反复跳跃阅读,在正文中也要对图表作必要的说明.\par
%	
%	对于预料之外的实验结果,必须首先小心证明其可靠性.读者只有在相信你的实验结果时才愿意花时间看你的分析.\par
%	必须用文字归纳整理出正式的实验结果或结论.可信的实验结果是课程报告最重要的内容.作为一个实验物理工作者,分析解释出错并不丢脸,实验结果不被采信则是致命的.\par
%	教学实验的结论往往是预先知道的. 所以,教师更关心的是你的说理过程. 一般说来,单由课内实验的结果不足以能得到明确的结论. 此时,你可以引用他人的研究结果来帮助帮助自己的论证,但必须注明出处. \par
%	确实不能得到明确结论时,可以给出几种可能结论并指出可以再做哪些实验来帮助作进一步的判断.\par
%	总之,分析讨论部分要做到: 论据要valid,论证要reasonable,结论要convincing.\par
%%%%%%%%%%%%%%%%%%%%%%%%%%%%%%

\section{结论}
%	首先要给出实验结果,然后再给出由实验结果分析得到的结果和结论.此部分给出的内容要比摘要中的全面,用词要更准确.\par
本实验中尝试使用熔石英制作的声光调制器对He-Ne激光的多个纵模进行锁模, 计算了所用激光腔的实际腔长. 虽然在调节光路上没有成功调节出锁模激光, 但本实验加深了对于声光调制锁模技术的理解, 并使实验者掌握了调节相关光路的方法, 从而为以后利用声光锁模技术进行实验提供了基础.
\section{思考题}
1.锁模用的声光调制器能用行波方式工作吗?为什么?
\par 答:不行, 因为行波模式产生拉曼-奈斯衍射的每一级都只存在一个频率, 因而无法形成锁模; 而驻波模式中衍射的每一级都含有多个频率成分, 因而可以锁模.

2.为什么要把声光调制器安放在尽量靠近谐振腔反射镜的一端?
\par 答:这样便于调节. 在有 M3 的时候可以直接利用 M1 和 M3 腔中形成的激光对声光调制器以及 M2 的方位进行调节.
\section{致谢}
%	此部分感谢同组人...和对实验和报告有帮助的人.
	感谢我的合作伙伴杨轩同学,他的工作是不可或缺的;感谢耐心的胡小永指导老师对我们的巨大帮助。
\begin{thebibliography}{99}
	\addcontentsline{toc}{section}{参考文献}
	\bibitem{ref1}北京大学物理学院光学所, 激光实验, 第二版, 北京: 北京大学物理学院, 2023.
\end{thebibliography}
\clearpage
\end{document}
