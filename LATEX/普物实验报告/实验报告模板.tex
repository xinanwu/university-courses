\documentclass[UTF8]{ctexart}
\usepackage{amsmath}
\usepackage{amssymb}
\usepackage{bm}
\usepackage{booktabs}
\usepackage{breqn}
\usepackage{color}
\usepackage{enumitem}
\usepackage{float}
\usepackage{geometry}
\usepackage{setspace}
\usepackage{hyperref}
\usepackage{titlesec}
\usepackage{graphicx,psfrag,epsfig}
\usepackage{caption}
\usepackage{amsmath,amsfonts,amssymb,amsthm,bm,upgreek}
\usepackage[mathscr]{eucal}
\usepackage{siunitx}
\usepackage{graphicx}
\usepackage{hyperref}
\usepackage{indentfirst}
\usepackage{multicol}
\usepackage{subfigure}
\usepackage{txfonts}
\usepackage{algorithm}
\usepackage{algorithmic}
\setlength{\parindent}{2em}
\usepackage{IEEEtrantools}
\usepackage{geometry}
\usepackage{listings}
\usepackage{lastpage}
\usepackage{tikz}
\usepackage{chngpage}
%\lstset{
%	commentstyle=\color{red!50!green!50!blue!50},%代码块背景色为浅灰色
%	rulesepcolor= \color{gray}, %代码块边框颜色
%	breaklines=true,  %代码过长则换行
%	numbers=left, %行号在左侧显示
%	numberstyle= \small,%行号字体
%	keywordstyle= \color{blue},%关键字颜色
%	frame=shadowbox,%用方框框住代码块
%	basicstyle=\ttfamily
%}
\definecolor{dkgreen}{rgb}{0,0.6,0}
\definecolor{mauve}{rgb}{0.9,0.1,0.4}
\definecolor{ash}{rgb}{0.8,0.8,0.8}
\lstset{ 
	language=Octave,                % the language of the code
	basicstyle=\ttfamily,           % the size of the fonts that are used for the code
	numbers=left,                   % where to put the line-numbers
	numberstyle=\small\color{gray},  % the style that is used for the line-numbers
	stepnumber=1,                   % the step between two line-numbers. If it's 1, each line
	% will be numbered
	numbersep=5pt,                  % how far the line-numbers are from the code
	backgroundcolor=\color{ash},      % choose the background color. You must add \usepackage{color}
	rulesepcolor= \color{gray}, %代码块边框颜色
	showspaces=false,               % show spaces adding particular underscores
	showstringspaces=false,         % underline spaces within strings
	showtabs=false,                 % show tabs within strings adding particular underscores
	frame=single,                   % adds a frame around the code
	rulecolor=\color{black},        % if not set, the frame-color may be changed on line-breaks within not-black text (e.g. commens (green here))
	tabsize=2,                      % sets default tabsize to 2 spaces
	captionpos=b,                   % sets the caption-position to bottom
	breaklines=true,                % sets automatic line breaking
	breakatwhitespace=false,        % sets if automatic breaks should only happen at whitespace
	title=\lstname,                   % show the filename of files included with \lstinputlisting;
	% also try caption instead of title
	frame=shadowbox,%用方框框住代码块
	keywordstyle=\color{blue},          % keyword style
	commentstyle=\color{dkgreen},       % comment style
	stringstyle=\color{mauve},         % string literal style
	escapeinside={\%*}{*)},            % if you want to add LaTeX within your code
	morekeywords={*,...}               % if you want to add more keywords to the set
}
\graphicspath{{figs/}}
\floatname{algorithm}{算法}  
\renewcommand{\algorithmicrequire}{\textbf{输入:}}  
\renewcommand{\algorithmicensure}{\textbf{输出:}} 
\author{
	吴熙楠}
\title{
	\heiti{普物实验报告}
}

\hypersetup{
	colorlinks=true,
	linkcolor=black
}


\begin{document}
	\maketitle

	\renewcommand{\contentsname}{目录} %将content转为目录
	\tableofcontents
	\newpage
		\renewcommand{\abstractname}{\large 摘要\\}
	\begin{abstract}
		 芜湖
			
	\textbf{关键词:}
	\end{abstract}
\section{实验目的}
\section{实验器材}
\section{实验过程及数据整理}
\section{分析与讨论}
\section{收获与感想}
\section{python源代码}
\begin{lstlisting}[language=python]
	import numpy as np
	import matplotlib.pyplot as plt
	filename = '三缝.txt'
	pos = []
	Efield = []
	list1=[]
	list2=[]
	with open(filename, 'r') as file_to_read:
	while True:
	lines = file_to_read.readline() # 整行读取数据
	if not lines:
	break
	pass
	p_tmp, E_tmp = [float(i) for i in lines.split()] 
	pos.append(p_tmp)  
	Efield.append(E_tmp)
	pass
	pos = np.array(pos) # 将数据从list类型转换为array类型。
	Efield = np.array(Efield)
	l=[[] for i in range(100)]
	m=361.14
	n=29.9066
	e=0.00064
	c=0.227733
	w=0.553067
	k=0
	g=0
	def residual_function(a,d):
	s=0
	i=0
	while i <=13999:
	s+=(Efield[i]-m*(((np.sin(a*(pos[i]-n)))/(a*(pos[i]-n)))**2)*(((np.sin(d*3*(pos[i]-n)))/(np.sin(d*(pos[i]-n))))**2))**2
	i+=1
	return s
	while k<=99:
	j=0
	while j<=99:
	l[k].append(residual_function(c+j*e,w+k*e))
	j+=1
	k+=1
	while g<=99:
	list1.append(min(l[g]))
	list2.append(l[g].index(min(l[g])))
	g+=1
	q=list1.index(min(list1))
	p=list2[q]
	print(p,q)
	x1=np.linspace(0,70,14000)
	y1=9*m*(((np.sin((c+p*e)*(x1-n)))/((c+p*e)*(x1-n)))**2)
	x2=np.linspace(0,70,14000)
	y2=m*(((np.sin((c+p*e)*(x2-n)))/((c+p*e)*(x2-n)))**2)*(((np.sin((w+q*e)*3*(x2-n)))/(np.sin((w+q*e)*(x2-n))))**2)
	plt.plot(pos,Efield,'orange')
	plt.plot(x2,y2,'black',linestyle='--')
	plt.plot(x1,y1,'r',linestyle='--')
	plt.xlabel('$\Delta x/mm$')
	plt.ylabel('$I$')
	plt.grid()
	plt.text(57,2880,'xScatter\n---FitLine\n---EnvelopingLine',size = 8,family = "fantasy",style = "italic",bbox = dict(alpha = 0.2))
	plt.show()      
\end{lstlisting}
	\begin{thebibliography}{99}  
		\bibitem{ref1}Pontryagin, L. S., V. G. Boltyanskii, R. V. Gamkrelize, and E. F. Mishchenko, The Mathematical
		Theory of Optimal Processes, New York, Wiley, 1962.
		\bibitem{ref2}Miller Neilan, Rachael \& Lenhart, Suzanne. (2010). An Introduction to Optimal Control with an Application in Disease Modeling. DIMACS Series in Discrete Mathematics. 75. 10.1090/dimacs/075/03. 
		\bibitem{ref3}J. R. Norris, Markov Chains. Cambridge;New York;: Cambridge University Press, 1997no. 2?.
		\bibitem{ref4}Gumel AB, Lenhart S (2010) Modeling paradigms and analysis of disease transmission models, vol 75. American Mathematical Society, Providence
	\end{thebibliography}
\end{document}