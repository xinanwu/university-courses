\documentclass[UTF8]{ctexart}
\usepackage{amsmath}
\usepackage{amssymb}
\usepackage{bm}
\usepackage{booktabs}
\usepackage{breqn}
\usepackage{color}
\usepackage{enumitem}
\usepackage{float}
\usepackage{graphicx}
\usepackage{hyperref}
\usepackage{indentfirst}
\usepackage{multicol}
\usepackage{ntheorem}
\usepackage{subfigure}
\usepackage{txfonts}
\usepackage{algorithm}
\usepackage{algorithmic}
\setlength{\parindent}{2em}
\usepackage{IEEEtrantools}
\usepackage{geometry}
\usepackage{listings}
\usepackage{lastpage}
\usepackage{tikz}
\usepackage{chngpage}
%\lstset{
%	commentstyle=\color{red!50!green!50!blue!50},%代码块背景色为浅灰色
%	rulesepcolor= \color{gray}, %代码块边框颜色
%	breaklines=true,  %代码过长则换行
%	numbers=left, %行号在左侧显示
%	numberstyle= \small,%行号字体
%	keywordstyle= \color{blue},%关键字颜色
%	frame=shadowbox,%用方框框住代码块
%	basicstyle=\ttfamily
%}
\definecolor{dkgreen}{rgb}{0,0.6,0}
\definecolor{mauve}{rgb}{0.9,0.1,0.4}
\definecolor{ash}{rgb}{0.8,0.8,0.8}
\lstset{ 
	language=Octave,                % the language of the code
	basicstyle=\ttfamily,           % the size of the fonts that are used for the code
	numbers=left,                   % where to put the line-numbers
	numberstyle=\small\color{gray},  % the style that is used for the line-numbers
	stepnumber=1,                   % the step between two line-numbers. If it's 1, each line
	% will be numbered
	numbersep=5pt,                  % how far the line-numbers are from the code
	backgroundcolor=\color{ash},      % choose the background color. You must add \usepackage{color}
	rulesepcolor= \color{gray}, %代码块边框颜色
	showspaces=false,               % show spaces adding particular underscores
	showstringspaces=false,         % underline spaces within strings
	showtabs=false,                 % show tabs within strings adding particular underscores
	frame=single,                   % adds a frame around the code
	rulecolor=\color{black},        % if not set, the frame-color may be changed on line-breaks within not-black text (e.g. commens (green here))
	tabsize=2,                      % sets default tabsize to 2 spaces
	captionpos=b,                   % sets the caption-position to bottom
	breaklines=true,                % sets automatic line breaking
	breakatwhitespace=false,        % sets if automatic breaks should only happen at whitespace
	title=\lstname,                   % show the filename of files included with \lstinputlisting;
	% also try caption instead of title
	frame=shadowbox,%用方框框住代码块
	keywordstyle=\color{blue},          % keyword style
	commentstyle=\color{dkgreen},       % comment style
	stringstyle=\color{mauve},         % string literal style
	escapeinside={\%*}{*)},            % if you want to add LaTeX within your code
	morekeywords={*,...}               % if you want to add more keywords to the set
}
\graphicspath{{figs/}}
\floatname{algorithm}{算法}  
\renewcommand{\algorithmicrequire}{\textbf{输入:}}  
\renewcommand{\algorithmicensure}{\textbf{输出:}} 
\author{北京大学物理学院\\
	吴熙楠}
\title{
	\heiti{霍尔效应实验报告}
}
\hypersetup{
	colorlinks=true,
	linkcolor=black
}


\begin{document}
	\maketitle
	\newtheorem{definition}{定义}[subsection]
	\newtheorem{function}{公式}[subsection]
	\newtheorem{summary}{小结}[subsection]
	\newtheorem{deduction}{推论}[subsection]
	\newtheorem{property}{性质}[subsection]
	\newtheorem{theo}{定理}[subsection]
	\newtheorem{step}{步骤}[subsection]
	\newtheorem{remark}{注记}[subsection]
	\newtheorem{proof}{证明}[subsection]
	\newenvironment{Theorem}[1][]{\par\noindent\textbf{定理}(#1)\quad}{\par}
	\newcommand{\rbra}[1]{\left( #1 \right)}
	\newcommand{\sbra}[1]{\left[ #1 \right]}
	\newcommand{\cbra}[1]{\left\{ #1 \right\}}
	\newcommand{\pbra}[1]{\left< #1 \right>}
	\newcommand{\abs}[1]{\left| #1 \right|}
	\newcommand{\fs}[2]{\displaystyle\frac{#1}{#2}}
	
	\newenvironment{myproof}{{\color{blue}证:}}
	
	\newenvironment{partlist}[1][]
	{\begin{enumerate}[itemsep=0pt, label=(\arabic*), wide, labelindent=\parindent, listparindent=\parindent, #1]}
		{\end{enumerate}}
	\renewcommand{\abstractname}{\Large 摘要\\}
	\begin{abstract}
		{\normalsize 固体材料中的载流子在外加磁场中运动时,因为受到洛仑兹力的作用而使轨迹发生偏移,并在材料两侧产生电荷积累,形成垂直于电流方向的电场,最终使载流子受到的洛仑兹力与电场斥力相平衡,从而在两侧建立起一个稳定的电势差即霍尔电压。这个就叫“霍尔效应”。在此次实验过程中,我们将学习霍尔效应测量磁场。
			
			\textbf{关键词:载流子、洛伦兹力、霍尔电压}}
	\end{abstract}
	
	\newpage
	\renewcommand{\contentsname}{目录} %将content转为目录
	\tableofcontents
	\newpage
	\section{实验目的}
	(1)了解霍尔效应的基本原理;
	\par (2)学习用霍尔效应测量磁场。
	\section{实验器材}
	霍尔效应仪,稳流电源,稳压电源,电流表,毫安表,功率函数发生器,特斯拉计,数字多用表,电阻箱,导线
	\section{实验过程及数据整理}
	\subsection{测量霍尔电流$I_{H}$和霍尔电压$U_{H}$的关系}
	\begin{center}
	$U_{H}=\dfrac{1}{4}(U_{1}-U_{2}+U_{3}-U_{4})$
    \end{center}
	\subsubsection{电流从$1,2$端流入,$I_{M}=0.600A$}
	\begin{table}[H]
		\caption{测量数据(电流从$1,2$端流入,$I_{M}=0.600A$)}
		\label{测量数据(电流从$1,2$端流入,$I_{M}=0.600A$)}
		\centering
		\begin{tabular}{c|*{5}{c}}
			\toprule[0.5mm]
			$I_{H}(mA)$&2.000&4.000&6.000&8.000&10.000\\
			\midrule
			$U_{1}(mV)$&4.31&8.61&13.17&17.27&21.78\\
			$U_{2}(mV)$&-4.34&-8.65&-13.25&-17.42&-22.02\\
			$U_{3}(mV)$&3.82&7.64&11.67&15.34&19.43\\
			$U_{4}(mV)$&-3.84&-7.66&-11.75&-15.45&-19.61\\
			\midrule
			$U_{H}(mV)$&4.08&8.14&12.46&16.37&20.71\\
			\bottomrule[0.5mm]
		\end{tabular}
	\end{table}
    \begin{figure}[H]
	\centering
	\caption{\label{1} 霍尔电流$I_{H}$和霍尔电压$U_{H}$的关系图}
	\includegraphics[width=12cm,height=9cm]  {iu1.png} 
	\end{figure}
\textbf{由图可知,霍尔电流$I_{H}$与霍尔电压$U_{H}$呈线性关系,相关系数$r=0.99998$也可以佐证其线性关系。}
	\subsubsection{电流从3,4端流入,$I_{M}=0.600A$}
	\begin{table}[H]
	\caption{测量数据(电流从$3,4$端流入,$I_{M}=0.600A$)}
	\label{测量数据(电流从$3,4$端流入,$I_{M}=0.600A$)}
	\centering
	\begin{tabular}{c|*{5}{c}}
		\toprule[0.5mm]
		$I_{H}(mA)$&2.000&4.000&6.000&8.000&10.000\\
		\midrule
		$U_{1}(mV)$&3.82&7.69&11.68&15.67&19.42\\
		$U_{2}(mV)$&-3.83&-7.73&-11.76&-15.81&-19.29\\
		$U_{3}(mV)$&4.31&8.65&13.11&17.46&21.77\\
        $U_{4}(mV)$&-4.32&-8.70&-13.20&-17.73&-21.96\\
		\midrule
		$U_{H}(mV)$&4.07&8.19&12.44&16.66&20.61\\
		\bottomrule[0.5mm]
	\end{tabular}
    \end{table}
    \begin{figure}[H]
	\centering
	\caption{\label{1} 霍尔电流$I_{H}$和霍尔电压$U_{H}$的关系图}
	\includegraphics[width=12cm,height=9cm]  {iu2.png} 
    \end{figure}
\textbf{由图可知,霍尔电流$I_{H}$与霍尔电压$U_{H}$呈线性关系,相关系数$r=0.999993$也可以佐证其线性关系。}
	\subsection{测量霍尔元件灵敏度$K_{H}$}
	\begin{table}[H]
	\caption{测量数据(电流从$1,2$端流入,$I_{H}=10.000mA$)}
	\label{测量数据(电流从$1,2$端流入,$I_{H}=10.000mA$)}
	\centering
	\begin{tabular}{c|*{11}{c}}
		\toprule[0.5mm]
		$I_{M}(A)$&0.000&0.100&0.200&0.300&0.400&0.500&0.600&0.700&0.800&0.900&1.000\\
		\midrule
		$B(mT)$&3.4&39.2&78.4&113.1&156.6&197.2&239.2&276.3&310.0&350.7&385.0\\
		\midrule
		$U_{1}(mV)$&1.22&4.40&7.60&10.94&14.42&17.76&21.20&24.42&28.42&31.81&34.65\\
		$U_{2}(mV)$&-1.42&-4.61&-7.78&-11.12&-14.61&-17.96&-21.41&-24.63&-28.63&-32.01&-34.82\\
		$U_{3}(mV)$&0.94&2.21&5.49&8.85&12.28&15.59&19.03&22.38&26.25&29.68&32.70\\
		$U_{4}(mV)$&-1.14&-2.43&-5.66&-9.03&-12.46&-15.78&-19.23&-22.59&-26.45&-29.92&-32.88\\
		\midrule
		$U_{H}(mV)$&1.18&3.41&6.63&9.99&13.44&16.77&20.22&23.51&27.44&30.86&33.76\\
		\bottomrule[0.5mm]
	\end{tabular}
    \end{table}
    \begin{figure}[H]
	\centering
	\caption{\label{1} $U_{H}-B$图}
	\includegraphics[width=12cm,height=9cm]  {bu.png} 
    \end{figure}
\textbf{由图可知,霍尔电压$U_{H}$与磁感应强度$B$呈线性关系,相关系数$r=0.9993$也可以佐证其线性关系。}
\par 考察斜率的不确定度,随机误差造成的部分为:
\begin{center}
	$\sigma_{k}=k\sqrt{\dfrac{1/r^{2}-1}{n-2}}=1.08\times10^{-3}T/A$
\end{center}
\par 我们认为霍尔电流$I_{H}$存在允差为$0.1mA$,则:
\begin{center}
	$\sigma_{I_{H}}=\dfrac{e}{\sqrt{3}}=0.0578mA$
\end{center}
\par 因此计算$K_{H}=\dfrac{k}{I_{H}}=8.66V/(T\cdot A)$
\par 不确定度为:
\begin{center}
	$\sigma_{K_{H}}=K_{H}\sqrt{(\dfrac{\sigma_{k}}{k})^{2}+(\dfrac{\sigma_{I_{H}}}{I_{H}})^{2}}=0.12V/(T\cdot A)$
\end{center}
\par \textbf{因此霍尔灵敏度$K_{H}=(8.66\pm 0.12)V/(T\cdot A)$}
	\subsection{做出磁化曲线图}
	我们由上一问求得的$K_{H}$可以通过$B=\dfrac{U_{H}}{K_{H}I_{H}}$计算出磁感应强度$B$的值:
	\begin{table}[H]
	\caption{$B-I_{M}$数据表}
	\label{$B-I_{M}$数据表}
	\centering
	\begin{tabular}{c|*{11}{c}}
		\toprule[0.5mm]
		$I_{M}(A)$&0.000&0.100&0.200&0.300&0.400&0.500&0.600&0.700&0.800&0.900&1.000\\
		\midrule
		$B(mT)$&13.6&39.4&76.5&115.3&155.2&193.6&233.4&271.4&316.8&356.3&389.8\\
		\bottomrule[0.5mm]
	\end{tabular}
    \end{table}
    \begin{figure}[H]
	\centering
	\caption{\label{1} 磁化曲线图}
	\includegraphics[width=12cm,height=9cm]  {b-i.png} 
    \end{figure}
	\subsection{测量电磁铁磁场的水平分布}
	\begin{table}[H]
	\caption{$B-x$数据表($I_{H}=10.000mA,I_{M}=0.600A$)}
	\label{$B-x$数据表($I_{H}=10.000mA,I_{M}=0.600A$)}
	\centering
	\begin{tabular}{c|*{4}{c}|c|c}
		\toprule[0.5mm]
        $x(mm)$&$U_{1}(mV)$&$U_{2}(mV)$&$U_{3}(mV)$&$U_{4}(mV)$&$U_{H}(mV)$&$B(mT)$\\
		\midrule
		40.00&21.41&-21.62&19.39&-19.57&20.50&236.7\\
		45.00&21.36&-21.57&19.33&-19.53&20.45&236.1\\
		50.00&21.53&-21.74&19.52&-19.72&20.63&238.2\\
		55.00&22.32&-22.54&20.20&-20.35&21.35&246.5\\		
		56.00&20.02&-20.23&17.90&-17.98&19.03&219.7\\
		57.00&15.61&-15.81&13.55&-13.72&14.67&169.4\\
		58.00&12.43&-12.63&10.40&-10.57&11.51&132.9\\
		59.00&10.44&-10.63&8.43&-8.62&9.53&110.0\\
		60.00&8.99&-9.17&7.00&-7.17&8.08&93.3\\
		\bottomrule[0.5mm]
	\end{tabular}
    \end{table}
    \begin{figure}[H]
	\centering
	\caption{\label{1} 电磁场水平分布(一半)}
	\includegraphics[width=10cm,height=7.6cm]  {b-x.png} 
    \end{figure}
\textbf{我们观察到磁感应强度$B$的值随着距离变化会在中心位置比较平缓,在边缘处下降比较陡峭。}
	\section{思考题}
	{\large{\textbf{Q}}}:\normalsize 在测量$B-I_{M}$曲线中,$I_{M}=0$时$U_{H}$测量端仍有较小的电压,这是为什么?
	\par{\large{\textbf{A}}}:\normalsize \textbf{因为$I_{M}=0$的时候会有众多的副效应,比如不等位效应,埃廷豪森效应,能斯特效应,里吉勒杜克效应,就算没有磁场$B$,也会存在电势差;而且有可能存在磁滞现象,即可能就算$I_{M}=0$,但磁感应强度$B$并不等于0。因此这样测量端仍然会有较小的电压。}
	\section{分析与讨论}
	(1){\large{\textbf{Q}}}:\normalsize 比较实验内容$4.1$中$(a)、(b)$两种接法观测的结果,并解释现象?
	\par{\large{\textbf{A}}}:\normalsize \textbf{我们交换接法后测量的结果发现$(a)$接法中$U_{1}$和$U_{2}$与$(b)$接法中$U_{3}$和$U_{4}$接近;$(a)$接法中$U_{3}$和$U_{4}$与$(b)$接法中$U_{1}$和$U_{2}$接近;但虽然接近,却不完全一致,因为会有很多的环境影响因此不会完全相同;而接近的原因是两种接法相当于电流方向反向,因此我们改变电流方向与改变接法是等效的,因此接近。}
	\par (2){\large{\textbf{Q}}}:\normalsize 说明实验内容$4.3$中为什么用计算的$B$作磁化曲线比用直接测量的$B$更好?
	\par{\large{\textbf{A}}}:\normalsize \textbf{因为我们直接测量的磁感应强度$B$只测量了一次,随机误差较大,可能对于线性比较不合理,而用计算的磁感应强度$B$完全反映线性结果,更适合拿来观察线性趋势。}
	\par (3){\large{\textbf{Q}}}:\normalsize 实验中观测到的各种曲线有什么主要特征,如何解释?
    \par{\large{\textbf{A}}}:\normalsize\textbf{(1)霍尔电流$I_{H}$和霍尔电压$U_{H}$线性正相关,因为霍尔电流$I_{H}$越大,则力平衡需要的电场越大,即霍尔电压$U_{H}$越大。(2)励磁电流$I_{M}$越大,磁感应强度$B$越大,霍尔电压$U_{H}$也越大,大致呈线性关系因为励磁电流越大,产生磁感应强度越大,同理力平衡需要的电场强度越大,即霍尔电压越大。(3)电磁铁磁场水平分布在中心最大,且变化比较平缓,然后改变较快,在边缘处大致变为中心磁感应强度的$\frac{1}{2}$,因为在中心处大概我们可以将其视作为无限长螺线管,在边缘处大致为半无限长螺线管,因此中心处磁场大致为边缘处磁场的$\frac{1}{2}$。} 
	\section{收获与感想}
    在我们本次实验中,我们通过老师的讲解,了解了霍尔效应的基本原理,同时通过使用活儿效应的仪器学习了使用霍尔效应测量磁场的方式。为我们以后将要从事凝聚态物理的同学学习量子霍尔效应,分数霍尔效应,量子反常霍尔效应等霍尔效应的衍生效应打下了基础,做好了铺垫。
\end{document}