
\documentclass[a4 paper,12pt]{article}
\usepackage[inner=2.0cm,outer=2.0cm,top=2.5cm,bottom=2.5cm]{geometry}
\usepackage{setspace}
\usepackage[rgb]{xcolor}
\usepackage{tabu}
\usepackage{multirow}
\usepackage{longtable}
\usepackage{graphicx}
\usepackage{verbatim}
\usepackage{longtable}
\usepackage{subcaption}
\usepackage{fancyhdr}
\usepackage[colorlinks=true, urlcolor=blue, linkcolor=blue, citecolor=blue]{hyperref}
\usepackage{booktabs}
\usepackage{amsmath,amsfonts,amsthm,amssymb}
\usepackage{setspace}
\usepackage{fancyhdr}
\usepackage{lastpage}
\usepackage{tikz}
\usetikzlibrary{positioning, arrows.meta}
\usepackage{extramarks}
\usepackage{ctex,amsmath,amsfonts,amssymb,bm,hyperref,graphicx}
\usepackage{chngpage}
\usepackage{soul,color}
\usepackage{graphicx,float,wrapfig}
\newcommand{\homework}[3]{
   \pagestyle{myheadings}
   \thispagestyle{plain}
   \newpage
   \setcounter{page}{1}
   \noindent
   \begin{center}
   \framebox{
        \vbox{\vspace{2mm}
        \hbox to 6.28in { {\bf 普物实验报告 \hfill} {\hfill {\rm #2} {\rm #3}} }
        \vspace{4mm}
        \hbox to 6.28in { {\Large \hfill #1  \hfill} }
        \vspace{3mm}}
   }
   \end{center}
   \vspace*{4mm}
}
\newcommand\numberthis{\addtocounter{equation}{1}\tag{\theequation}}

\begin{document}
\homework{分光计的调节与使用}{1900011413}{吴熙楠}
\tableofcontents
\newpage
\begin{abstract}
	分光计是使光按波长分散兼供光学测量的仪器。一般由准直管、棱镜台和望远镜3种主要部件构成。可用于测量波长、棱镜角、棱镜材料的折射率和色散率等,具有很高的测量精度。本实验中我们将学会分光计的调节与使用,并用掠入射法和最小偏向角法测量三棱镜的折射率,并测定玻璃材料的色散曲线。\\
	\par\textbf{关键词: 折射率,色散,最小偏向角}

\end{abstract}

\section{实验目的}
\noindent
(1)了解分光计的结构和工作原理,掌握其调节要求和使用规范。\\
(2)用分光计测定三棱镜的顶角。\\
(3)用掠入射和最小偏向角法测定棱镜折射率。\\
(4)测定玻璃材料的色散曲线。
\section{实验仪器}
分光计、三棱镜、钠灯、汞灯、平面镜、毛玻璃、放大镜
\section{实验数据处理}
\subsection{自准直法测定棱镜的顶角}
\begin{table}[H]
	\centering
	\caption{三棱镜顶角的测量}
	\label{三棱镜顶角的测量}
	\begin{tabular}{c|*{4}{c}}
		\toprule[0.5mm]
		i&$\phi_{1}$&$\phi_{1}^{\prime}$&$\phi_{2}$&$\phi_{2}^{\prime}$\\
		\midrule
		1&$64^{\circ}42^{\prime}$&$244^{\circ}44^{\prime}$&$124^{\circ}42^{\prime}$&$304^{\circ}45^{\prime}$\\
		2&$64^{\circ}42^{\prime}$&$244^{\circ}45^{\prime}$&$124^{\circ}43^{\prime}$&$304^{\circ}46^{\prime}$\\
		3&$64^{\circ}42^{\prime}$&$244^{\circ}44^{\prime}$&$124^{\circ}43^{\prime}$&$304^{\circ}46^{\prime}$\\
		\bottomrule[0.5mm]
	\end{tabular}
\end{table}
对于上述数据我们有(注意因为游标盘转过了零点所以应该加上$360^{\circ}$):
\begin{center}
	$\bar{A}=\dfrac{1}{3}\sum\limits_{i=1}^{3}[180^{\circ}-\dfrac{1}{2}(\phi_{1}+\phi_{1}^{\prime}-\phi_{2}-\phi_{2}^{\prime}+360^{\circ})]=60^{\circ}1^{\prime}$
\end{center}
\par 所以对于顶角$A$的$A$类误差(测量三次导致的随机误差)有:
\begin{center}
	$\sigma_{A,a}=\sqrt{\dfrac{\sum\limits_{i=1}^{3}(A_{i}-\bar{A})^{2}}{n-1}}=0.5^{\prime}$
\end{center}
\par 对于顶角$A$的$B$类误差(取系统允差为$1^{\prime}$)有:
\begin{center}
	$\sigma_{A,b}=\dfrac{\Delta}{\sqrt{3}}=0.577^{\prime}$
\end{center}
\par 合成不确定度为:
\begin{center}
	$\sigma_{A}=\sqrt{\sigma_{A,a}^{2}+\sigma_{A,b}^{2}}=0.763^{\prime}$
\end{center}
\par 因此顶角$A$为(因为不确定度计算小于$1^{\prime}$,取不确定度为最小分度值$1^{\prime}$):
\begin{center}
	$A=60^{\circ}1^{\prime}\pm 1^{\prime}$
\end{center}
\subsection{掠入射法测量三棱镜折射率}
\begin{table}[H]
	\centering
	\caption{掠入射法测量棱镜折射率数据表}
	\label{掠入射法测量棱镜折射率数据表}
	\begin{tabular}{c|*{4}{c}}
	\toprule[0.5mm]
	i&$\phi_{1}$&$\phi_{1}^{\prime}$&$\phi_{2}$&$\phi_{2}^{\prime}$\\
	\midrule
	1&$123^{\circ}14^{\prime}$&$303^{\circ}16^{\prime}$&$81^{\circ}47^{\prime}$&$261^{\circ}49^{\prime}$\\
	2&$123^{\circ}14^{\prime}$&$303^{\circ}15^{\prime}$&$81^{\circ}47^{\prime}$&$261^{\circ}49^{\prime}$\\
	3&$123^{\circ}16^{\prime}$&$303^{\circ}16^{\prime}$&$81^{\circ}47^{\prime}$&$261^{\circ}50^{\prime}$\\
	\bottomrule[0.5mm]
    \end{tabular}
\end{table}
对于上述数据我们有:
\begin{center}
	$\bar{\phi}=\dfrac{1}{3}\sum\limits_{i=1}^{3}\dfrac{1}{2}(\phi_{1}+\phi_{1}^{\prime}-\phi_{2}-\phi_{2}^{\prime})=41^{\circ}27^{\prime}$
\end{center}
\par 代入掠入射折射率公式可得:
\begin{center}
	$n=\sqrt{(\dfrac{sin\phi+cosA}{sinA})^{2}+1}=1.6730$
\end{center}
\par 对于偏向角$\phi$的$A$类误差(测量三次导致的随机误差)有:
\begin{center}
	$\sigma_{\phi,A}=\sqrt{\dfrac{\sum\limits_{i=1}^{3}(\phi_{i}-\bar{\phi})^{2}}{n-1}}=0.5^{\prime}$
\end{center}
\par 对于偏向角$\phi$的$B$类误差(取系统允差为$1^{\prime}$)有:
\begin{center}
	$\sigma_{\phi,B}=\dfrac{\Delta}{\sqrt{3}}=0.577^{\prime}$
\end{center}
\par 合成不确定度为:
\begin{center}
	$\sigma_{\phi}=\sqrt{\sigma_{\phi,A}^{2}+\sigma_{\phi,B}^{2}}=0.763^{\prime}$
\end{center}
\par 对于折射率的不确定度为:
\begin{center}
	$\sigma_{n}=\sqrt{(1-\dfrac{1}{n^{2}})[(\dfrac{1+cosAsin\phi}{sin^{2}A}\sigma_{A})^{2}+(\dfrac{cos\phi}{sinA}\sigma_{\phi})^{2}]}=4\times10^{-4}$
\end{center}
\par 因此掠入射法测量棱镜折射率为:
\begin{center}
	$n=1.6730\pm 0.0004$
\end{center}
\subsection{最小偏向角法测量三棱镜折射率(绿光)}
\begin{table}[H]
	\centering
	\caption{最小偏向角法测量棱镜折射率数据表}
	\label{最小偏向角法测量棱镜折射率数据表}
	\begin{tabular}{c|*{4}{c}}
		\toprule[0.5mm]
		i&$\phi_{1}$&$\phi_{1}^{\prime}$&$\phi_{2}$&$\phi_{2}^{\prime}$\\
		\midrule
		1&$120^{\circ}21^{\prime}$&$300^{\circ}20^{\prime}$&$66^{\circ}18^{\prime}$&$246^{\circ}19^{\prime}$\\
		2&$120^{\circ}23^{\prime}$&$300^{\circ}21^{\prime}$&$66^{\circ}18^{\prime}$&$246^{\circ}20^{\prime}$\\
		3&$120^{\circ}21^{\prime}$&$300^{\circ}20^{\prime}$&$66^{\circ}19^{\prime}$&$246^{\circ}19^{\prime}$\\
		\bottomrule[0.5mm]
	\end{tabular}
\end{table}
对于上述数据我们有:
\begin{center}
	$\bar{\delta_{m}}=\dfrac{1}{3}\sum\limits_{i=1}^{3}\dfrac{1}{2}(\phi_{1}+\phi_{1}^{\prime}-\phi_{2}-\phi_{2}^{\prime})=54^{\circ}3^{\prime}$
\end{center}
\par 代入最小偏向角折射率公式可得:
\begin{center}
	$n=\dfrac{sin\dfrac{\delta_{m}+A}{2}}{sin\dfrac{A}{2}}=1.6781$
\end{center}
\par 对于最小偏向角$\delta_{m}$的$A$类误差(测量三次导致的随机误差)有:
\begin{center}
	$\sigma_{\delta_{m},A}=\sqrt{\dfrac{\sum\limits_{i=1}^{3}(\delta_{m}-\bar{\delta_{m}})^{2}}{n-1}}=1.5^{\prime}$
\end{center}
\par 对于最小偏向角$\delta_{m}$的$B$类误差(取系统允差为$1^{\prime}$)有:
\begin{center}
	$\sigma_{\delta_{m},B}=\dfrac{\Delta}{\sqrt{3}}=0.577^{\prime}$
\end{center}
\par 合成不确定度为:
\begin{center}
	$\sigma_{\delta_{m}}=\sqrt{\sigma_{\delta_{m},A}^{2}+\sigma_{\delta_{m},B}^{2}}=1.61^{\prime}$
\end{center}
\par 对于折射率的不确定度为:
\begin{center}
	$\sigma_{n}=\dfrac{\sqrt{\left(\dfrac{sin\dfrac{\delta_{m}}{2}}{sin\dfrac{A}{2}}\sigma_{A}\right)^{2}+\left(cos\dfrac{\delta_{m}+A}{2}\sigma_{\delta_{m}}\right)^{2}}}{2sin\dfrac{A}{2}}=3\times10^{-4}$
\end{center}
\par 因此最小偏向角法测量棱镜折射率为:
\begin{center}
	$n=1.6781\pm 0.0003$
\end{center}
\subsection{测定玻璃材料的色散曲线}
\begin{table}[H]
	\centering
	\caption{汞灯不同波长测量数据表}
	\label{汞灯不同波长测量数据表}
	\begin{tabular}{c|*{6}{c}}
		\toprule[0.5mm]
		$\lambda/nm$&$\phi_{1}$&$\phi_{1}^{\prime}$&$\phi_{2}$&$\phi_{2}^{\prime}$&$\delta_{m}$&$n$\\
		\midrule
	    579.07&$119^{\circ}39^{\prime}$&$299^{\circ}43^{\prime}$&$66^{\circ}00^{\prime}$&$246^{\circ}03^{\prime}$&$53^{\circ}39^{\prime}$&1.6737\\
	    576.96&$119^{\circ}40^{\prime}$&$299^{\circ}44^{\prime}$&$66^{\circ}00^{\prime}$&$246^{\circ}03^{\prime}$&$53^{\circ}40^{\prime}$&1.6739\\
	    546.07&$120^{\circ}23^{\prime}$&$300^{\circ}21^{\prime}$&$66^{\circ}18^{\prime}$&$246^{\circ}20^{\prime}$&$54^{\circ}3^{\prime}$&1.6781\\
	    491.60&$121^{\circ}00^{\prime}$&$301^{\circ}03^{\prime}$&$66^{\circ}03^{\prime}$&$246^{\circ}6^{\prime}$&$54^{\circ}57^{\prime}$&1.6860\\
	    435.84&$121^{\circ}43^{\prime}$&$301^{\circ}46^{\prime}$&$65^{\circ}21^{\prime}$&$245^{\circ}24^{\prime}$&$56^{\circ}22^{\prime}$&1.6992\\
	    407.78&$120^{\circ}57^{\prime}$&$300^{\circ}58^{\prime}$&$63^{\circ}36^{\prime}$&$243^{\circ}38^{\prime}$&$57^{\circ}21^{\prime}$&1.7083\\
	    404.66&$122^{\circ}36^{\prime}$&$302^{\circ}39^{\prime}$&$64^{\circ}59^{\prime}$&$245^{\circ}1^{\prime}$&$57^{\circ}38^{\prime}$&1.7107\\
	    \bottomrule[0.5mm]
	\end{tabular}
\end{table}
其中:$\delta_{m}=\dfrac{1}{2}(\phi_{1}+\phi_{1}^{\prime}-\phi_{2}-\phi_{2}^{\prime}),n=\dfrac{sin\dfrac{\delta_{m}+A}{2}}{sin\dfrac{A}{2}}$\\
\par 将上表数据用柯西公式$n=A+\dfrac{B}{\lambda^{2}}+\dfrac{C}{\lambda^{4}}$拟合可得:
\begin{figure}[H]
	\centering
    \includegraphics[width=11cm,height=8cm]  {三棱镜色散曲线.png} 
    \caption{\label{1} 三棱镜色散曲线}
\end{figure}
\section{分析与讨论}
\begin{center}
	误差来源分析
\end{center}
\par 在分光计的调节与使用的实验过程中,误差主要来源于以下几个方面:\textcircled{1}读数时存在的随机误差;\textcircled{2}调节分光计时载物平台或者是望远镜未调节好所带来的系统误差;\textcircled{3}分光计本身制作时所带有的仪器允差;\textcircled{4}最小偏向角法中用肉眼来判断最小偏向角所带来的误差;\textcircled{5}最小偏向角法中如暗绿谱线以及暗紫谱线由于难以观察导致的最小偏向角判断的误差;\textcircled{6}在测量三棱镜折射率时螺丝未扭紧导致仪器未固定移动的误差;\textcircled{7}狭缝过宽导致的谱线中心定位不准的误差。
\section{收获与感想}
分光计是一种能够精确测量角度的仪器,可用于测量波长、棱镜角、棱镜材料的折射率和色散率等,拥有广泛的应用。我们在本节课学习了分光计的结构和工作原理,掌握其调节要求和使用规范,同时我们对于分光计做了简单的应用,测量了三棱镜的色散曲线,同时我们对于分光计测量三棱镜折射率的不确定度进行了计算,定量了解到分光计测量的精确性,为以后的光学实验打好良好的基础。
\end{document} 
