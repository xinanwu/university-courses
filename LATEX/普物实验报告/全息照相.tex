\documentclass[UTF8]{ctexart}
\usepackage{amsmath}
\usepackage{amssymb}
\usepackage{bm}
\usepackage{booktabs}
\usepackage{breqn}
\usepackage{color}
\usepackage{enumitem}
\usepackage{float}
\usepackage{graphicx}
\usepackage{hyperref}
\usepackage{indentfirst}
\usepackage{multicol}
\usepackage{ntheorem}
\usepackage{subfigure}
\usepackage{txfonts}
\usepackage{algorithm}
\usepackage{algorithmic}
\setlength{\parindent}{2em}
\usepackage{IEEEtrantools}
\usepackage{geometry}
\usepackage{listings}
\usepackage{lastpage}
\usepackage{tikz}
\usepackage{chngpage}
%\lstset{
%	commentstyle=\color{red!50!green!50!blue!50},%代码块背景色为浅灰色
%	rulesepcolor= \color{gray}, %代码块边框颜色
%	breaklines=true,  %代码过长则换行
%	numbers=left, %行号在左侧显示
%	numberstyle= \small,%行号字体
%	keywordstyle= \color{blue},%关键字颜色
%	frame=shadowbox,%用方框框住代码块
%	basicstyle=\ttfamily
%}
\definecolor{dkgreen}{rgb}{0,0.6,0}
\definecolor{mauve}{rgb}{0.9,0.1,0.4}
\definecolor{ash}{rgb}{0.8,0.8,0.8}
\lstset{ 
	language=Octave,                % the language of the code
	basicstyle=\ttfamily,           % the size of the fonts that are used for the code
	numbers=left,                   % where to put the line-numbers
	numberstyle=\small\color{gray},  % the style that is used for the line-numbers
	stepnumber=1,                   % the step between two line-numbers. If it's 1, each line
	% will be numbered
	numbersep=5pt,                  % how far the line-numbers are from the code
	backgroundcolor=\color{ash},      % choose the background color. You must add \usepackage{color}
	rulesepcolor= \color{gray}, %代码块边框颜色
	showspaces=false,               % show spaces adding particular underscores
	showstringspaces=false,         % underline spaces within strings
	showtabs=false,                 % show tabs within strings adding particular underscores
	frame=single,                   % adds a frame around the code
	rulecolor=\color{black},        % if not set, the frame-color may be changed on line-breaks within not-black text (e.g. commens (green here))
	tabsize=2,                      % sets default tabsize to 2 spaces
	captionpos=b,                   % sets the caption-position to bottom
	breaklines=true,                % sets automatic line breaking
	breakatwhitespace=false,        % sets if automatic breaks should only happen at whitespace
	title=\lstname,                   % show the filename of files included with \lstinputlisting;
	% also try caption instead of title
	frame=shadowbox,%用方框框住代码块
	keywordstyle=\color{blue},          % keyword style
	commentstyle=\color{dkgreen},       % comment style
	stringstyle=\color{mauve},         % string literal style
	escapeinside={\%*}{*)},            % if you want to add LaTeX within your code
	morekeywords={*,...}               % if you want to add more keywords to the set
}
\graphicspath{{figs/}}
\floatname{algorithm}{算法}  
\renewcommand{\algorithmicrequire}{\textbf{输入:}}  
\renewcommand{\algorithmicensure}{\textbf{输出:}} 
\author{
	吴熙楠}
\title{
	\heiti{全息照相实验报告}
}

\hypersetup{
	colorlinks=true,
	linkcolor=black
}


\begin{document}
	\maketitle
	\newtheorem{definition}{定义}[subsection]
	\newtheorem{function}{公式}[subsection]
	\newtheorem{summary}{小结}[subsection]
	\newtheorem{deduction}{推论}[subsection]
	\newtheorem{property}{性质}[subsection]
	\newtheorem{theo}{定理}[subsection]
	\newtheorem{step}{步骤}[subsection]
	\newtheorem{remark}{注记}[subsection]
	\newtheorem{proof}{证明}[subsection]
	\newenvironment{Theorem}[1][]{\par\noindent\textbf{定理}(#1)\quad}{\par}
	\newcommand{\rbra}[1]{\left( #1 \right)}
	\newcommand{\sbra}[1]{\left[ #1 \right]}
	\newcommand{\cbra}[1]{\left\{ #1 \right\}}
	\newcommand{\pbra}[1]{\left< #1 \right>}
	\newcommand{\abs}[1]{\left| #1 \right|}
	\newcommand{\fs}[2]{\displaystyle\frac{#1}{#2}}
	
	\newenvironment{myproof}{{\color{blue}证:}}
	
	\newenvironment{partlist}[1][]
	{\begin{enumerate}[itemsep=0pt, label=(\arabic*), wide, labelindent=\parindent, listparindent=\parindent, #1]}
		{\end{enumerate}}
	
	\renewcommand{\contentsname}{目录} %将content转为目录
	\tableofcontents
	\newpage
	\renewcommand{\abstractname}{\large 摘要\\}
	\begin{abstract}
		全息照片是一种记录被摄物体反射(或透射)光波中全部信息的先进照相技术。全息照相的原理是依据光的干涉原理,利用两束光的干涉记录被摄物体的信息,并通过衍射重现被摄物体信息。激光束被分光镜一分为二,其中一束照到被拍摄的景物上,被称物光束;另一束直接照到感光胶片即全息干板上,称为参考光束。当物光束被物体反射后,其反射光束也照射在胶片上,就完成了全息照片的摄制过程,彩虹全息照相技术还可拍立体图像。
		
		\textbf{关键词:干涉,衍射,激光束}
	\end{abstract}
	\section{实验目的}
	(1)了解全息照相的基本原理;
	\par (2)学习全息照相的实验技术,拍摄合格的全息图;
	\par (3)了解摄影暗室技术。
	\section{实验器材}
	光学平台,He-Ne激光器及电源,快门及定时曝光器,扩束透镜,反射镜和分束器,光功率计,全息底片,被摄物体,显微镜,暗室技术及使用设备。
	\section{实验过程及数据整理}
	\subsection{实验光路}
	\begin{figure}[H]
		\centering
		\includegraphics[width=10cm,height=7cm]  {记录光路.png} 
		\caption{\label{1}全息记录光路图}
	\end{figure}
\begin{figure}[H]
	\centering
	\includegraphics[width=10cm,height=7cm]  {再现光路.png} 
		\caption{\label{1}全息再现光路图}
\end{figure}
\par 参考光光程$L_{1}=109.0cm$,物光光程$L_{2}=107.5cm$,物光与参考光夹角$\alpha \approx 25^{\circ}$,曝光时间$t=4.0s$
\subsection{实验现象记录及分析}
\textbf{1、原位观察虚像}
\par (1)虚像位于原物处,与原物基本等大,正立,如图3所示;(2)从底片不同位置观察,像基本不变;(3)旋转底片,改变入射角,像的大小未变,发生转动,且转动方向与底片转动方向相反,同时光强发生变化,当入射角增大到某一个值时像将会消失;(4)将底片向远离透镜方向移动,像变大,反之向靠近透镜方向移动,像变小。\textbf{原因:再现时照明光为发散球面波,等效于在底片后放入一球心在球面波球心的凹透镜。移远底片相当于增大凹透镜焦距,由放大率公式$\beta=\frac{1}{1+|s/f|}$可知,横向放大率也会增大;反之靠近的话像会变小。}
\begin{figure}[H]
	\centering
	\includegraphics[width=7cm,height=8cm]  {叶子.png} 
	\caption{\label{1}原位观察虚像图}
\end{figure}
\par (当我们参考光使用会聚光时,我们观察到的像为原像关于胶板法线的等大镜面对称像,如图4所示,旋转底片,改变入射角,像的大小未变,发生转动,且转动方向与底片转动方向相反,同时因为底片在虚光源与透镜中间,所以将底片向远离透镜方向移动,像变小,反之向靠近透镜方向移动,像变大,原因同上。)
\begin{figure}[H]
	\centering
	\includegraphics[width=7cm,height=7cm]  {叶子2.png} 
	\caption{\label{1}共轭光观察虚像图}
\end{figure}
\par \textbf{2、另一个像有没有?位置?实像还是虚像?为什么?如果没有,请分析原因。}
\par 未观察到另一个像,原因为:参考光、照明光均为发散球面波,对于-1级衍射像,其像距计算可为$v=\frac{uR}{R-2u}$,$u$为物距,$R$为参考光球心。观察不到像说明成像在较远处,即$u\sim \frac{R}{2}$,且光强较小。
\par \textbf{3、底片翻转 180 度,用共轭光照射,观察实像,记录实像位置,大小,正倒}
\par 共轭光照射将会形成放大正立的实像,其位置如图所示:
\begin{figure}[H]
	\centering
	\includegraphics[width=10cm,height=7cm]  {共轭光.png} 
	\caption{\label{1}共轭光照射实像位置图}
\end{figure}
\par \textbf{4、用激光直接照射来再现,像有什么特点,请与共轭光再现做对比并分析其原因。}
\par 当激光从底片背面正入射时,可在法线两侧分别看到放大正立的实像与很弱的虚像;当激光从底片正面正入射时,仅仅在法线一侧看到放大正立的虚像。如图所示:
\begin{figure}[H]
	\centering
	\includegraphics[width=10cm,height=7cm]  {激光背射.png} 
	\caption{\label{1}激光背射光路图}
\end{figure}
\begin{figure}[H]
	\centering
	\includegraphics[width=10cm,height=7cm]  {激光正射.png} 
	\caption{\label{1}激光正射光路图}
\end{figure}
\par 原因:因为参考光为发散球面波,等效于在底片处的一凸透镜,使得在原物后成一放大正立的像(虚实由光入射方向决定),在以底片为镜像的另一侧有一同样放大正立的共轭像。激光直接正入射时,照明光为平面波,故最终成像如图6,7所示。而共轭光再现的图5中,照明光为会聚球面波,等效于在底片处的一凸透镜,这使得经过参考光等效的凸透镜所成的像再次被放大,仍为正立放大的像。
	\section{收获与感想}
	在本次实验中我们了解了全息照相的基本原理,学习了全息照相的实验技术,并拍摄了合格的全息图,同时也了解了摄影暗室技术,同时也为我们一年的普物实验课程画上了圆满的句号!
\end{document}