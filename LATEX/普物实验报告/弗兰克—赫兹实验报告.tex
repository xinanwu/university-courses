
\documentclass[a4 paper,12pt]{article}
\usepackage[inner=2.0cm,outer=2.0cm,top=2.5cm,bottom=2.5cm]{geometry}
\usepackage{setspace}
\usepackage[rgb]{xcolor}
\usepackage{tabu}
\usepackage{multirow}
\usepackage{longtable}
\usepackage{graphicx}
\usepackage{verbatim}
\usepackage{longtable}
\usepackage{subcaption}
\usepackage{fancyhdr}
\usepackage[colorlinks=true, urlcolor=blue, linkcolor=blue, citecolor=blue]{hyperref}
\usepackage{booktabs}
\usepackage{amsmath,amsfonts,amsthm,amssymb}
\usepackage{setspace}
\usepackage{fancyhdr}
\usepackage{lastpage}
\usepackage{tikz}
\usetikzlibrary{positioning, arrows.meta}
\usepackage{extramarks}
\usepackage{ctex,amsmath,amsfonts,amssymb,bm,hyperref,graphicx}
\usepackage{chngpage}
\usepackage{soul,color}
\usepackage{graphicx,float,wrapfig}
\newcommand{\homework}[3]{
   \pagestyle{myheadings}
   \thispagestyle{plain}
   \newpage
   \setcounter{page}{1}
   \noindent
   \begin{center}
   \framebox{
        \vbox{\vspace{2mm}
        \hbox to 6.28in { {\bf 普物实验报告 \hfill} {\hfill {\rm #2} {\rm #3}} }
        \vspace{4mm}
        \hbox to 6.28in { {\Large \hfill #1  \hfill} }
        \vspace{3mm}}
   }
   \end{center}
   \vspace*{4mm}
}
\newcommand\numberthis{\addtocounter{equation}{1}\tag{\theequation}}

\begin{document}
\homework{弗兰克—赫兹实验}{1900011413}{吴熙楠}
\tableofcontents
\newpage
\begin{abstract}
	弗兰克—赫兹实验证明原子内部结构存在分立的定态能级.这个事实直接证明了汞原子具有玻尔所设想的那种“完全确定的,互相分立的能量状态”,是对玻尔的原子量子化模型的第一个决定性的证据.我们将用弗兰克——赫兹管(充汞或氩),F—H管电源,扫描电源和微电流放大器,电炉及控温器观测汞和氩原子的第一激发态曲线,验证原子能级的存在.\\
	\par\textbf{关键词:原子能级,分立,激发态}

\end{abstract}

\section{实验目的}
\noindent
(1)了解弗兰克—赫兹用伏安法证明原子能级存在的原理和方法.\\
(2)学习用伏安法测量非线性元件.\\
(3)学习微电流的测量.

\section{实验仪器}
弗兰克——赫兹管(充汞或氩),F—H管电源(三组直流电源,供灯丝和各栅极的偏压),扫描电源和微电流放大器(DC:0$\~{}90V$,电流范围$10^{-8},10^{-9}A$),电炉及控温器.

\section{实验数据处理}

\subsection{Ar管弗兰克—赫兹实验}

\begin{table}[H]
	\caption{Ar第一激发态峰值电压与峰间距数据表}
	\centering
	\begin{tabular}{*{7}{c}}
		\toprule[0.5mm]
		$U_{kg2}(V)$&17.0&28.0&39.9&52.5&65.5&79.0\\
		\midrule
		$\Delta U(V)$&/&11.0&11.9&12.6&13.0&13.5\\
		\bottomrule[0.5mm]
	\end{tabular}
\end{table}
其中:$U_{HH}=2.5V,U_{G_{2}A}=6.5V,U_{G_{1}k}=2.0V$

\subsubsection{Ar第一激发态的管流与加速电压曲线}
\begin{figure}[H] 
	\centering
	\caption{\label{1} Ar第一激发态的管流与加速电压曲线}
	\includegraphics[width=13cm,height=10cm]  {Ar.png} 
\end{figure}

\subsubsection{Ar第一激发电位与n的关系}
\begin{figure}[H] 
	\centering
	\caption{\label{1} Ar第一激发电位与n的关系}
	\includegraphics[width=13cm,height=10cm]  {Ar线性拟合.png} 
\end{figure}
\noindent
A类误差计算:
\begin{center}
	$\sigma_{U_{1}A}=U_{1}\sqrt{\dfrac{\frac{1}{r^{2}}-1}{n-2}}\sqrt{\sum\limits_{i=1}^{n}(x_{i}-\bar{x})^{2}}=0.9012V$
\end{center}
取仪器允差为:
\begin{center}
	$\Delta=0.5V$
\end{center}
B类误差计算:
\begin{center}
	$\sigma_{U_{1}B}=\dfrac{\Delta}{\sqrt{3}}=0.2887V$
\end{center}
合成误差为:
\begin{center}
	$\sigma=\sqrt{\sigma_{U_{1}A}^{2}+\sigma_{U_{1}B}^{2}}=0.9463V$
\end{center}
合成不确定度为:
\begin{center}
	$\sigma_{U_{1}}=\dfrac{\sigma}{\sqrt{\sum\limits_{i=1}^{n}(x_{i}-\bar{x})^{2}}}=0.2V$
\end{center}
因此:
\begin{center}
	$U_{1}=(12.1\pm0.2)V$
\end{center}

\subsection{Hg管弗兰克—赫兹实验}

\begin{table}[H]
	\caption{Hg第一激发态峰值电压与峰间距数据表}
	\centering
	\begin{tabular}{*{7}{c}}
		\toprule[0.5mm]
		$U_{kg2}(V)$&5.2&9.9&14.8&19.8&24.6&29.8\\
		\midrule
		$\Delta U(V)$&/&4.7&4.9&5.0&4.8&5.2\\
		\bottomrule[0.5mm]
	\end{tabular}
\end{table}
其中:$U_{1}=1.0V,U_{3}=2.0V,\theta=176.0^{\circ}C$

\subsubsection{Hg第一激发态的管流与加速电压的关系}
\begin{figure}[H] 
	\centering
	\caption{\label{1} Hg第一激发态的管流与加速电压曲线}
	\includegraphics[width=13cm,height=10cm]  {Hg.png} 
\end{figure}

\subsubsection{Hg第一激发电位与n的关系}
\begin{figure}[H] 
	\centering
	\caption{\label{1} Hg第一激发电位与n的关系}
	\includegraphics[width=13cm,height=10cm]  {Hg线性拟合.png} 
\end{figure}
\noindent
A类误差计算:
\begin{center}
	$\sigma_{U_{1}A}=U_{1}\sqrt{\dfrac{\frac{1}{r^{2}}-1}{n-2}}\sqrt{\sum\limits_{i=1}^{n}(x_{i}-\bar{x})^{2}}=0.1300V$
\end{center}
取仪器允差为:
\begin{center}
	$\Delta=0.2V$
\end{center}
B类误差计算:
\begin{center}
	$\sigma_{U_{1}B}=\dfrac{\Delta}{\sqrt{3}}=0.1155V$
\end{center}
合成误差为:
\begin{center}
	$\sigma=\sqrt{\sigma_{U_{1}A}^{2}+\sigma_{U_{1}B}^{2}}=0.174V$
\end{center}
合成不确定度为:
\begin{center}
	$\sigma_{U_{1}}=\dfrac{\sigma}{\sqrt{\sum\limits_{i=1}^{n}(x_{i}-\bar{x})^{2}}}=0.04V$
\end{center}
因此:
\begin{center}
	$U_{1}=(4.92\pm0.04)V$
\end{center}

\section{分析与讨论}
\subsection{$U_{3}=3V$时Hg第一激发态的管流与加速电压的关系}
\begin{figure}[H] 
	\centering
	\caption{\label{1} Hg第一激发态的管流与加速电压曲线}
	\includegraphics[width=10cm,height=7cm]  {Hg1.png} 
\end{figure}
\subsection{$U_{3}=1V$时Hg第一激发态的管流与加速电压的关系}
\begin{figure}[H] 
	\centering
	\caption{\label{1} Hg第一激发态的管流与加速电压曲线}
	\includegraphics[width=10cm,height=7cm]  {Hg2.png} 
\end{figure}
\noindent
根据数据我们对最后两个峰值进行指数拟合:$U_{out}=Ae^{bU_{3}}$\\
计算结果为:
\begin{center}
	$A=701.92mV,\quad b=-0.556V^{-1},\quad r=0.9942$
\end{center}
可见峰值电压$U_{out}$与反向电压$U_{3}$之间接近于指数函数衰减的关系,但由于实验数据不够多,因此与实际情况还是有一点偏差,但和预期的$U_{3}$越大$U_{out}$越小一致,因为减速电压越大,打到P板上电子的密度n与速度v就越小,则输出电压越小.
\subsection{实验中测得曲线的特征}
实验测的曲线总是起伏上升的.当KG间的电压逐渐增大时电子在KG间加速得到越来越多的能量,当电子得到的能量较低时,如果此时与原子碰撞还不足以影响汞原子的内部能量,板极电流I随GK间电压增加而增加.当KG间加速电压达原子第一激发电势时,电子与原子碰撞,将自己的能量传递给原子,原子从基态被激发到第一激发态.而电子失去几乎全部动能,电子将不能克服拒斥场而达板极P,P处的电流开始下降.继续升高加速电压即GK间电压,电子获得的动能亦有所增加,这时电子即使碰撞损失大量的能量,但还留有部分电子可以到达P处,因而P处电流又开始回升,当KG间的电压是两倍的原子激发电势时,电子在KG间有可能经过两次碰撞失去能量,因此又造成极板电流也就是P处电流下降.
\subsection{分析测量第一激发电位时误差的主要来源}
我们测量第一激发电位时,主要的测量误差来自于读数时的随机误差以及仪器的允差,因为测量时仪器读数不稳定以及改变加速电压值后仪器有一段响应时间,我们测量时未达到稳态就开始测量,这均会造成读书时因为时间不同而造成的读数误差,还有就是仪器允差也造成了我们的测量误差.
\section{原始数据整理}
\subsection{Ar管弗兰克—赫兹实验原始数据表}
\begin{footnotesize}
	\centering
\begin{longtable}{||r|r||r|r||r|r||r|r||r|r||}
	\caption{Ar管弗兰克—赫兹实验原始数据表} 
	\label{Ar管弗兰克—赫兹实验原始数据表}\\
	\hline
	\hline
	$U_{k_{2}g}/V$&$I_{p}/10^{-9}A$&$U_{k_{2}g}/V$&$I_{p}/10^{-9}A$&$U_{k_{2}g}/V$&$I_{p}/10^{-9}A$&$U_{k_{2}g}/V$&$I_{p}/10^{-9}A$&$U_{k_{2}g}/V$&$I_{p}/10^{-9}A$\\
	\hline
	\hline
		0.5   & 0     & 18.5  & 25.7  & 38    & 66    & 52.8  & 99.4  & 69.5  & 89.9 \\
		1     & 0     & 19    & 21.9  & 38.2  & 67.6  & 53    & 99    & 70    & 86 \\
		1.5   & 0     & 19.5  & 17.8  & 38.4  & 69.3  & 53.5  & 95.9  & 70.5  & 84.4 \\
		2     & 0     & 20    & 12.9  & 38.6  & 71.1  & 54    & 91.8  & 71    & 85.1 \\
		2.5   & 0     & 20.5  & 9     & 38.8  & 71.7  & 54.5  & 85.5  & 71.5  & 88.4 \\
		3     & 0     & 21    & 6.8   & 39    & 72.5  & 55    & 77    & 72    & 92.6 \\
		3.5   & 0     & 21.5  & 4.8   & 39.2  & 73.4  & 55.5  & 68.6  & 72.5  & 99.5 \\
		4     & 0     & 22    & 4     & 39.4  & 73.8  & 56    & 59.2  & 73    & 106.9 \\
		4.5   & 0     & 22.5  & 3.7   & 39.6  & 74.3  & 56.5  & 50.4  & 73.5  & 114 \\
		5     & 0     & 23    & 5.3   & 39.8  & 74.6  & 57    & 43.5  & 74    & 120.9 \\
		5.5   & 0     & 23.5  & 9.2   & 40    & 74.6  & 57.5  & 39    & 74.5  & 128 \\
		6     & 0     & 24    & 16    & 40.2  & 74.5  & 58    & 40.5  & 75    & 135.6 \\
		6.5   & 0     & 24.5  & 23    & 40.4  & 74.2  & 58.5  & 45.2  & 75.5  & 143.1 \\
		7     & 0     & 25    & 32.8  & 41    & 71.2  & 59    & 51.4  & 76    & 148.2 \\
		7.5   & 0     & 25.5  & 37.8  & 41.5  & 67.6  & 59.5  & 59.8  & 76.2  & 150.4 \\
		8     & 0     & 26    & 44.3  & 42    & 61.7  & 60    & 69.2  & 76.4  & 153.2 \\
		8.5   & 0     & 26.5  & 47.4  & 42.5  & 54    & 60.5  & 77.6  & 76.6  & 155.6 \\
		9     & 0.4   & 27    & 50.2  & 43    & 44.6  & 61    & 87    & 76.8  & 157.3 \\
		9.5   & 1.3   & 27.2  & 51.5  & 43.5  & 35.3  & 61.5  & 94.5  & 77    & 159.5 \\
		10    & 2.2   & 27.4  & 52    & 44    & 27    & 62    & 104.5 & 77.2  & 161.2 \\
		10.5  & 3.6   & 27.6  & 52.2  & 44.5  & 19    & 62.5  & 109   & 77.4  & 163 \\
		11    & 5.6   & 27.8  & 52.6  & 45    & 13.9  & 62.7  & 114.3 & 77.6  & 164.5 \\
		11.5  & 8.6   & 28    & 52.8  & 45.5  & 10.9  & 62.9  & 114.8 & 77.8  & 165.5 \\
		12    & 15    & 28.2  & 52.6  & 46    & 13.2  & 63.1  & 116.4 & 78    & 166.6 \\
		12.5  & 19.5  & 28.4  & 52.2  & 46.5  & 22.2  & 63.3  & 118.4 & 78.2  & 167.6 \\
		13    & 23.2  & 28.6  & 51.8  & 47    & 31.8  & 63.5  & 120.3 & 78.4  & 168.4 \\
		13.5  & 25.3  & 28.8  & 51.2  & 47.5  & 42.2  & 63.7  & 122.5 & 78.6  & 169 \\
		14    & 26.2  & 29    & 50.2  & 48    & 54.9  & 63.9  & 124.2 & 78.8  & 169.5 \\
		14.5  & 26.6  & 29.5  & 47    & 48.5  & 63.4  & 64.1  & 125.9 & 79    & 169.8 \\
		15    & 27    & 30    & 41.3  & 49    & 71.3  & 64.3  & 126.9 & 79.2  & 169.6 \\
		15.2  & 27.1  & 30.5  & 35.5  & 49.5  & 78.6  & 64.5  & 128.2 & 79.4  & 169.6 \\
		15.4  & 27.2  & 31    & 27.3  & 50    & 84.6  & 64.7  & 129.1 & 79.6  & 168.9 \\
		15.6  & 28.3  & 31.5  & 20.2  & 50.2  & 87.6  & 64.9  & 129.9 & 79.8  & 168.8 \\
		\hline
		\hline
		15.8  & 28.9  & 32    & 13    & 50.4  & 89    & 65.1  & 130.5 & 80    & 167.8 \\
		16    & 29.3  & 32.5  & 8.1   & 50.6  & 90.6  & 65.3  & 130.7 & 80.5  & 164.6 \\
		16.2  & 29.6  & 33    & 5     & 50.8  & 92.5  & 65.5  & 130.8 & 81    & 160.8 \\
		16.4  & 29.7  & 33.5  & 3.2   & 51    & 94.4  & 65.7  & 130.6 & 81.5  & 156.8 \\
		16.6  & 29.9  & 34    & 3.2   & 51.2  & 95.7  & 65.9  & 130.3 & 82    & 152.8 \\
		16.8  & 30    & 34.5  & 5.9   & 51.4  & 97    & 66    & 129.9 & 82.5  & 149.8 \\
		17    & 30.1  & 35    & 16    & 51.6  & 98    & 66.5  & 127   & 83    & 147.8 \\
		17.2  & 30.1  & 35.5  & 26.2  & 51.8  & 98.6  & 67    & 122.7 & 83.5  & 146.6 \\
		17.4  & 29.9  & 36    & 37.3  & 52    & 99.2  & 67.5  & 116.8 & 84    & 146.8 \\
		17.6  & 29.6  & 36.5  & 44.6  & 52.2  & 99.6  & 68    & 110   & 84.5  & 148.6 \\
		17.8  & 29.2  & 37    & 54.4  & 52.4  & 99.8  & 68.5  & 102.1 & 85    & 152.3 \\
		18    & 28.2  & 37.5  & 61.3  & 52.6  & 99.8  & 69    & 95.2  &       &  \\
		\hline
	\end{longtable}
\end{footnotesize}
\subsection{Hg管弗兰克—赫兹实验原始数据表}
\begin{center}
\begin{longtable}{||r|r||r|r||r|r||r|r||}
	\caption{Hg管弗兰克—赫兹实验原始数据表} 
    \label{Hg管弗兰克—赫兹实验原始数据表}\\
    \hline
    \hline
    $U_{k_{2}g}/V$&$U_{out}/mV$&    $U_{k_{2}g}/V$&$U_{out}/mV$&    $U_{k_{2}g}/V$&$U_{out}/mV$&    $U_{k_{2}g}/V$&$U_{out}/mV$\\
    \hline
    \hline
		0.3   & 4.7   & 8.4   & 35.3  & 15.9  & 34    & 24    & 192 \\
		0.6   & 5.3   & 8.7   & 47.7  & 16.2  & 23.2  & 24.1  & 196 \\
		0.9   & 4.8   & 9     & 69.8  & 16.5  & 16.7  & 24.2  & 207.7 \\
		1.2   & 5.1   & 9.3   & 87.1  & 16.8  & 15.4  & 24.3  & 213.7 \\
		1.5   & 5.4   & 9.6   & 104   & 17.1  & 19.5  & 24.4  & 232.4 \\
		1.8   & 5.5   & 9.7   & 112   & 17.4  & 26.5  & 24.5  & 235.8 \\
		2.1   & 5.8   & 9.8   & 116.2 & 17.7  & 47.1  & 24.6  & 239.2 \\
		2.4   & 5.8   & 9.9   & 118.6 & 18    & 57.6  & 24.7  & 238.7 \\
		2.7   & 6.2   & 10    & 112.6 & 18.3  & 84.8  & 24.8  & 227 \\
		3     & 7     & 10.1  & 101.2 & 18.6  & 114   & 24.9  & 222.1 \\
		3.3   & 7.8   & 10.2  & 88.5  & 18.9  & 162.5 & 25.2  & 186.6 \\
		3.6   & 10.5  & 10.5  & 57.1  & 19.2  & 182.4 & 25.5  & 152.6 \\
		3.9   & 13.3  & 10.8  & 21.8  & 19.3  & 186.7 & 25.8  & 116.3 \\
		4.2   & 17.6  & 11.1  & 18.2  & 19.4  & 187.5 & 26.1  & 83.5 \\
		4.5   & 21.7  & 11.4  & 12.1  & 19.5  & 194   & 26.4  & 56.3 \\
		4.6   & 24.5  & 11.7  & 11.3  & 19.6  & 196.3 & 26.7  & 41.3 \\
		4.7   & 26.6  & 12    & 13.2  & 19.7  & 197.4 & 27    & 34.7 \\
		4.8   & 28.2  & 12.3  & 17.5  & 19.8  & 199.2 & 27.3  & 49.1 \\
		4.9   & 32    & 12.6  & 26    & 19.9  & 190.3 & 27.6  & 60.5 \\
		5     & 33.8  & 12.9  & 37.6  & 20    & 175.4 & 27.9  & 76.7 \\
		5.1   & 35    & 13.2  & 53.6  & 20.1  & 164.9 & 28.2  & 105.6 \\
		\hline
		\hline
		5.2   & 35.6  & 13.5  & 73.4  & 20.4  & 125.5 & 28.5  & 142 \\
		5.3   & 35.3  & 13.8  & 101.6 & 20.7  & 76.5  & 28.8  & 180.1 \\
		5.4   & 33.8  & 14.1  & 128.3 & 21    & 56.8  & 28.9  & 194 \\
		5.5   & 30.3  & 14.4  & 163   & 21.3  & 35.5  & 29    & 196.9 \\
		5.6   & 25.6  & 14.5  & 163.8 & 21.6  & 27.7  & 29.1  & 200.8 \\
		5.7   & 17.8  & 14.6  & 173.2 & 21.9  & 27.6  & 29.2  & 213.2 \\
		6     & 11.7  & 14.7  & 174.2 & 22.2  & 29.8  & 29.3  & 227.5 \\
		6.3   & 9.2   & 14.8  & 176.1 & 22.5  & 41.2  & 29.4  & 238.7 \\
		6.6   & 8.5   & 14.9  & 176.2 & 22.8  & 58.7  & 29.5  & 239.5 \\
		6.9   & 8.4   & 15    & 154.8 & 23.1  & 85.2  & 29.6  & 242.5 \\
		7.2   & 9.3   & 15.1  & 132.4 & 23.4  & 121.5 & 29.7  & 246.7 \\
		7.5   & 12    & 15.2  & 118   & 23.7  & 145.5 & 29.8  & 254.4 \\
		7.8   & 16    & 15.3  & 109.1 & 23.8  & 169.4 & 29.9  & 252.6 \\
		8.1   & 25.2  & 15.6  & 71.3  & 23.9  & 179.4 & 30    & 245.6 \\
		\hline
\end{longtable}
\end{center}
\subsection{$U_{3}=3V$时Hg弗兰克—赫兹实验原始数据表}
\begin{longtable}{||r|r||r|r||r|r||}
	\caption{$U_{3}=3V$时Hg弗兰克—赫兹实验原始数据表}\\
	\hline
	\hline
	$U_{k_{2}g}/V$&$U_{out}/mV$&	$U_{k_{2}g}/V$&$U_{out}/mV$&	$U_{k_{2}g}/V$&$U_{out}/mV$\\
	\hline
	\hline
		20.1  & 78.7  & 24.3  & 99.1  & 27.3  & 21.3 \\
		20.4  & 57.1  & 24.4  & 106.1 & 27.6  & 23 \\
		20.7  & 41    & 24.5  & 110.6 & 27.9  & 26.4 \\
		21    & 26    & 24.6  & 116.5 & 28.2  & 34.3 \\
		21.3  & 21    & 24.7  & 121.5 & 28.5  & 49.6 \\
		21.6  & 15.3  & 24.8  & 116.5 & 28.8  & 70.3 \\
		21.9  & 11.8  & 24.9  & 111.3 & 29.2  & 106.7 \\
		22.2  & 12.7  & 25    & 106.7 & 29.3  & 111 \\
		22.5  & 12.6  & 25.3  & 95.7  & 29.4  & 116 \\
		22.8  & 16.4  & 25.6  & 82.8  & 29.5  & 123 \\
		23.1  & 27.4  & 25.9  & 64.7  & 29.6  & 126.5 \\
		23.4  & 33.8  & 26.2  & 47.8  & 29.7  & 130 \\
		23.7  & 53.1  & 26.5  & 36.2  & 29.8  & 133.7 \\
		24    & 80.4  & 26.9  & 27.4  & 29.9  & 129.7 \\
		24.2  & 94    & 27    & 25    & 30    & 126.7 \\
		\hline
\end{longtable}
\subsection{$U_{3}=1V$时Hg弗兰克—赫兹实验原始数据表}
\begin{longtable}{||r|r||r|r||r|r||}
	\caption{$U_{3}=1V$时Hg弗兰克—赫兹实验原始数据表}\\
	\hline
	\hline
	$U_{k_{2}g}/V$&$U_{out}/mV$&	$U_{k_{2}g}/V$&$U_{out}/mV$&	$U_{k_{2}g}/V$&$U_{out}/mV$\\
	\hline
	\hline
		21    & 72.2  & 24.4  & 349.1 & 27.9  & 201.3 \\
		21.3  & 60.6  & 24.5  & 351.9 & 28.2  & 248.2 \\
		21.6  & 63.2  & 24.6  & 344.9 & 28.5  & 296.7 \\
		21.9  & 74    & 24.7  & 338.1 & 28.8  & 346.7 \\
		22.2  & 98.3  & 25    & 296.6 & 29.1  & 372 \\
		22.5  & 116.5 & 25.3  & 241.5 & 29.2  & 388.7 \\
		22.8  & 151.2 & 25.6  & 181.7 & 29.3  & 396.4 \\
		23.1  & 185.2 & 25.9  & 145.6 & 29.4  & 412.3 \\
		23.4  & 228.1 & 26.2  & 116.5 & 29.5  & 417.3 \\
		23.7  & 270   & 26.5  & 105.7 & 29.6  & 421.7 \\
		24    & 324   & 26.8  & 110.3 & 29.7  & 408.2 \\
		24.1  & 339.8 & 27    & 119.2 & 29.8  & 396 \\
		24.2  & 345.7 & 27.3  & 144.1 & 29.9  & 389.2 \\
		24.3  & 355.6 & 27.6  & 175   & 30    & 375.4 \\
		\hline
\end{longtable}
\end{document} 
