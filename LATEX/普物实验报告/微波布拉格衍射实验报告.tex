\documentclass[UTF8]{ctexart}
\usepackage{amsmath}
\usepackage{amssymb}
\usepackage{bm}
\usepackage{booktabs}
\usepackage{breqn}
\usepackage{color}
\usepackage{enumitem}
\usepackage{float}
\usepackage{graphicx}
\usepackage{hyperref}
\usepackage{indentfirst}
\usepackage{multicol}
\usepackage{ntheorem}
\usepackage{subfigure}
\usepackage{txfonts}
\usepackage{algorithm}
\usepackage{algorithmic}
\setlength{\parindent}{2em}
\usepackage{IEEEtrantools}
\usepackage{geometry}
\usepackage{listings}
\usepackage{lastpage}
\usepackage{tikz}
\usepackage{chngpage}
%\lstset{
%	commentstyle=\color{red!50!green!50!blue!50},%代码块背景色为浅灰色
%	rulesepcolor= \color{gray}, %代码块边框颜色
%	breaklines=true,  %代码过长则换行
%	numbers=left, %行号在左侧显示
%	numberstyle= \small,%行号字体
%	keywordstyle= \color{blue},%关键字颜色
%	frame=shadowbox,%用方框框住代码块
%	basicstyle=\ttfamily
%}
\definecolor{dkgreen}{rgb}{0,0.6,0}
\definecolor{mauve}{rgb}{0.9,0.1,0.4}
\definecolor{ash}{rgb}{0.8,0.8,0.8}
\lstset{ 
	language=Octave,                % the language of the code
	basicstyle=\ttfamily,           % the size of the fonts that are used for the code
	numbers=left,                   % where to put the line-numbers
	numberstyle=\small\color{gray},  % the style that is used for the line-numbers
	stepnumber=1,                   % the step between two line-numbers. If it's 1, each line
	% will be numbered
	numbersep=5pt,                  % how far the line-numbers are from the code
	backgroundcolor=\color{ash},      % choose the background color. You must add \usepackage{color}
	rulesepcolor= \color{gray}, %代码块边框颜色
	showspaces=false,               % show spaces adding particular underscores
	showstringspaces=false,         % underline spaces within strings
	showtabs=false,                 % show tabs within strings adding particular underscores
	frame=single,                   % adds a frame around the code
	rulecolor=\color{black},        % if not set, the frame-color may be changed on line-breaks within not-black text (e.g. commens (green here))
	tabsize=2,                      % sets default tabsize to 2 spaces
	captionpos=b,                   % sets the caption-position to bottom
	breaklines=true,                % sets automatic line breaking
	breakatwhitespace=false,        % sets if automatic breaks should only happen at whitespace
	title=\lstname,                   % show the filename of files included with \lstinputlisting;
	% also try caption instead of title
	frame=shadowbox,%用方框框住代码块
	keywordstyle=\color{blue},          % keyword style
	commentstyle=\color{dkgreen},       % comment style
	stringstyle=\color{mauve},         % string literal style
	escapeinside={\%*}{*)},            % if you want to add LaTeX within your code
	morekeywords={*,...}               % if you want to add more keywords to the set
}
\graphicspath{{figs/}}
\floatname{algorithm}{算法}  
\renewcommand{\algorithmicrequire}{\textbf{输入:}}  
\renewcommand{\algorithmicensure}{\textbf{输出:}} 
\author{
	吴熙楠}
\title{
	\heiti{微波布拉格衍射实验报告}
}

\hypersetup{
	colorlinks=true,
	linkcolor=black
}


\begin{document}
	\maketitle
	\newtheorem{definition}{定义}[subsection]
	\newtheorem{function}{公式}[subsection]
	\newtheorem{summary}{小结}[subsection]
	\newtheorem{deduction}{推论}[subsection]
	\newtheorem{property}{性质}[subsection]
	\newtheorem{theo}{定理}[subsection]
	\newtheorem{step}{步骤}[subsection]
	\newtheorem{remark}{注记}[subsection]
	\newtheorem{proof}{证明}[subsection]
	\newenvironment{Theorem}[1][]{\par\noindent\textbf{定理}(#1)\quad}{\par}
	\newcommand{\rbra}[1]{\left( #1 \right)}
	\newcommand{\sbra}[1]{\left[ #1 \right]}
	\newcommand{\cbra}[1]{\left\{ #1 \right\}}
	\newcommand{\pbra}[1]{\left< #1 \right>}
	\newcommand{\abs}[1]{\left| #1 \right|}
	\newcommand{\fs}[2]{\displaystyle\frac{#1}{#2}}
	
	\newenvironment{myproof}{{\color{blue}证:}}
	
	\newenvironment{partlist}[1][]
	{\begin{enumerate}[itemsep=0pt, label=(\arabic*), wide, labelindent=\parindent, listparindent=\parindent, #1]}
		{\end{enumerate}}
	
	\renewcommand{\contentsname}{目录} %将content转为目录
	\tableofcontents
	\newpage
	\renewcommand{\abstractname}{\large 摘要\\}
	\begin{abstract}
		晶体有规则的几何形状,晶体中原子按规则排列组成晶格,对于晶体的散射而言,同一层的散射线,当散线与晶面间的夹角等于掠射角时,在这个方向上射线产生相长干涉,而对于不同层的散射线,当光程差为波长的整数倍时,各个面的散射线相互加强,形成光强的极大,即布拉格衍射。我们在本次实验中将会采用模拟晶体进行微波布拉格衍射实验,从而证明布拉格公式。
		
		\textbf{关键词:晶体,散射,布拉格公式}
	\end{abstract}
	\section{实验目的}
	(1)了解并学习微波器件的使用;
	\par (2)了解布拉格衍射原理,利用微波在模拟晶体上的衍射验证布拉格公式。
	\section{实验器材}
	微波分光仪,模拟晶体,单缝,反射板(两块),分束板
	\section{实验过程及数据整理}
	\subsection{模拟晶体布拉格衍射}
	\subsubsection{[100]面模拟布拉格衍射}
	\begin{table}[H]
		\centering
		\caption{[100]面模拟晶体布拉格衍射数据表}
		\begin{tabular}{||r|r||r|r||r|r||r|r||}
			\toprule[0.5mm]
			$\theta/^{\circ}$     & $I/\mu A$     & $\theta/^{\circ}$     & $I/\mu A$     & $\theta/^{\circ}$     & $I/\mu A$     & $\theta/^{\circ}$     & $I/\mu A$ \\
			\midrule
			90    & 44    & 69    & 62    & 55    & 5     & 37    & 20 \\
			89    & 64    & 68    & 49    & 50    & 6     & 36    & 22 \\
			88    & 87    & 67    & 26    & 45    & 4     & 35    & 15 \\
			87    & 98    & 66    & 20    & 44    & 6     & 34    & 10 \\
			86    & 100   & 65    & 19    & 43    & 8     & 33    & 6 \\
			85    & 87    & 64    & 12    & 42    & 8     & 32    & 5 \\
			84    & 52    & 63    & 7     & 41    & 10    & 31    & 6 \\
			80    & 4     & 62    & 4     & 40    & 12    & 30    & 10 \\
			75    & 7     & 61    & 3     & 39    & 14    & 25    & 2 \\
			70    & 30    & 60    & 2     & 38    & 18    & -     &-  \\
			\bottomrule[0.5mm]
		\end{tabular}
	\end{table}
\begin{figure}[H]
	\centering
	\includegraphics[width=10cm,height=7cm]  {100.png} 
	\caption{\label{1}[100]面模拟布拉格衍射}
\end{figure}
\par 从图中我们可以看到有三个峰,其中前两个是我们需要的,而最后一个峰是因为微波在不同晶面的透射导致的增强,因此我们只计入前两个峰的角度,$\theta_{1}=37^{\circ},\theta_{2}=69^{\circ}$
\par 其中我们按照理论公式计算出的理论值为:$\theta_{1}=36.9^{\circ},\theta_{2}=66.4^{\circ}$,可见实际上我们模拟晶体的布拉格衍射还算较为准确(在可以允许的误差范围内),但由于转动角度不能准确读出以及初始位置没有准确对准等原因,与理论值还是有一些偏差。
	\subsubsection{[110]面模拟布拉格衍射}
\begin{table}[H]
	\centering
	\caption{[110]面模拟晶体布拉格衍射数据表}
	\begin{tabular}{||r|r||r|r||r|r||}
		\toprule[0.5mm]
		$\theta/^{\circ}$     & $I/\mu A$     & $\theta/^{\circ}$     & $I/\mu A$     & $\theta/^{\circ}$     & $I/\mu A$    \\
		\midrule
		90    & 52    & 80    & 42    & 54    & 14 \\
		89    & 54    & 75    & 12    & 53    & 11 \\
		88    & 58    & 70    & 2     & 52    & 6 \\
		87    & 58    & 65    & 1     & 51    & 4 \\
		86    & 64    & 60    & 8     & 50    & 4 \\
		85    & 77    & 59    & 14    & 45    & 0 \\
		84    & 90    & 58    & 18    & 40    & 0 \\
		83    & 90    & 57    & 20    & 35    & 0 \\
		82    & 78    & 56    & 18    & 30    & 0 \\
		81    & 66    & 55    & 15    & 25    & 0 \\
		\bottomrule[0.5mm]
	\end{tabular}
\end{table}	
\begin{figure}[H]
	\centering
	\includegraphics[width=10cm,height=7cm]  {110.png} 
	\caption{\label{1}[110]面模拟布拉格衍射}
\end{figure}
\par 从图中我们可以看到有两个峰,其中第一个是我们需要的,而第二个峰是因为微波在不同晶面的透射导致的增强,因此我们只计入第一个峰的角度,$\theta=57^{\circ}$
\par 其中我们按照理论公式计算出的理论值为:$\theta=55.5^{\circ}$,可见实际上我们模拟晶体的布拉格衍射还算较为准确(在可以允许的误差范围内),但由于转动角度不能准确读出以及初始位置没有准确对准等原因,与理论值还是有一些偏差。
	\subsection{单缝衍射}
	\begin{table}[H]
		\centering
		\caption{单缝衍射数据表}
		\begin{tabular}{||r|r||r|r||r|r||r|r||}
			\toprule[0.5mm]
			$\theta/^{\circ}$     & $I/\mu A$     & $\theta/^{\circ}$     & $I/\mu A$     & $\theta/^{\circ}$     & $I/\mu A$     & $\theta/^{\circ}$     & $I/\mu A$ \\
			\midrule
			0     & 82    & 24    & 2     & -10   & 42    & -27   & 1 \\
			5     & 68    & 26    & 1     & -15   & 24    & -28   & 0 \\
			10    & 45    & 27    & 0     & -20   & 8     & -29   & 0 \\
			15    & 22    & 28    & 0     & -22   & 6     & -30   & 0 \\
			20    & 4     & 30    & 0     & -24   & 4     & -      &-  \\
			22    & 3     & -5    & 72    & -26   & 2     &  -     & - \\
			\bottomrule[0.5mm]
		\end{tabular}
	\end{table}
\begin{figure}[H]
	\centering
	\includegraphics[width=10cm,height=7cm]  {单缝衍射.png} 
	\caption{\label{1}模拟单缝衍射图}
\end{figure}
\par 我们通过测量数据可得:微波在衍射第一次达到极小值的时候$\theta\approx\dfrac{28^{\circ}+27^{\circ}}{2}=27.5^{\circ}\Rightarrow \lambda=asin\theta=3.23cm$
\par 我们比较微波波长的理论值$\lambda=\dfrac{c}{f}=3.20cm$可得,我们利用单缝衍射的方法测得的微波波长也是较为准确的。
	\subsection{迈克尔逊干涉}
	\begin{table}[H]
		\centering
		\caption{迈克尔逊干涉数据表}
		\begin{tabular}{|r|rrrr|}
			\toprule[0.5mm]
			$n$&1&2&3&4\\
			\midrule
			$d/mm$&59.962&43.417&26.965&11.785\\
			\bottomrule[0.5mm]
		\end{tabular}
	\end{table}
\begin{figure}[H]
	\centering
	\includegraphics[width=10cm,height=7cm]  {迈干.png} 
	\caption{\label{1}模拟迈克尔逊干涉实验数据图}
\end{figure}
\par 我们从图中作图可得斜率为:$\dfrac{\lambda}{2}=16.098mm\Rightarrow \lambda=32.2mm$
\par 而实际上我们计算理论值可得$\lambda=\dfrac{c}{f}=32.02mm$,可见我们用迈克尔逊干涉的方法测出的微波波长还是较为准确的。
	\section{分析与讨论}
	\textbf{为何在模拟晶体布拉格衍射实验中入射角在$80^{\circ}+$时会出现一个衍射峰?}
	\par \textbf{答:因为入射角超过80$^{\circ}$时,因为已经接近垂直入射,而模拟晶体中间间隙为空的,因此微波的透射被接收器接受到的几率增大,因此接收到的电流强度会增大,但因为模拟晶体的小球有一定的体积大小,故如果完全垂直入射,则必然会导致微波被小球180$^{\circ}$反射的几率增大,接受到的电流减小,故两种效应叠加将会在$80^{\circ}-90^{\circ}$范围内形成一个峰位。}
	\section{收获与感想}
	在本次实验中,我们了解并学习了微波器件的使用,了解了布拉格衍射原理,同时我们利用微波在模拟晶体上的衍射验证了布拉格公式。布拉格衍射为我们学习晶体以及光学中的重要知识,本次实验给我们拓宽了眼界同时从实验方面了解了此公式的意义。
	
\end{document}